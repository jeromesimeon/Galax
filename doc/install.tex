This chapter contains the following sections:
\begin{itemize}
\item Prerequisites (Section~\ref{sec:prelim}) : prerequisites for
  Galax installation.
\item Source Distribution (Section~\ref{sec:src-dist}) :  
  Installation instructions for the source distribution.
\eat{
\item Binary Distribution (Section~\ref{sec:bin-dist}) : requirements and
  installation instructions for the binary distribution.
\item Operating-System Installation Notes (Section~\ref{sec:os-notes})
  : detailed  installation notes for each supported target.
}
\item Running the W3C XQuery Test Suite (XQTS)
(Section~\ref{sec:testsuite}).
\item Web-form interface (Section~\ref{sec:website}) : installation
instructions for Galax's web-form interface.
\end{itemize}

\section{Prerequisites}
\label{sec:prelim}
\begin{itemize}
\item PCRE 5.0 or higher is required. If you do not have PCRE version
  5.0 or higher\footnote{\ahrefurl{http://www.pcre.org/}} installed.
\item (Optional) If you plan to use Jungle, Galax's secondary storage
  manager, you will need Berkeley DB version 4.4 or
  higher\footnote{\ahrefurl{http://dev.sleepycat.com/downloads/releasehistorybdb.html}}
  installed.
\end{itemize}

\section{Source Installation}
\label{sec:src-dist}

There are two methods for installing Galax from source:

\begin{enumerate}
\item \textbf{RECOMMENDED METHOD:} Use GODI installation and configuration tool
     to automatically install Galax and all the libraries on which it
     depends.

\item Install Galax and the libraries on which it depends manually.

This method may be preferable if you are an OCaml user, already have
an installation of OCaml and its libraries, and do not want to install
a new version of OCaml.  Even if you are already an OCaml user, the
first method is always recommended.
\end{enumerate}

\subsection{
Installing Galax using GODI installation and configuration tool}

\textbf{Warning:} Installing packages using GODI is an ON-LINE process, so
preferably, you should be connected to the Net by a fast, hard-wire
connection.

\begin{enumerate}
\item Download GODI from : \ahrefurl{http://godi.ocaml-programming.de/}

Follow GODI installation instructions through "bootstrap\_stage2"
command, adding the\\\verb+<godi-prefix-path>/bin+ and
\verb+<godi-prefix-path>/sbin+ directories to your PATH as instructed.

\textbf{Note:} For OCaml users, after GODI bootstrapping phase, the
OCaml executables are located in \verb+<godi-prefix-path>/bin+ and the
OCaml libraries are located in \verb+<godi-prefix-path>/lib/ocaml+.

\item Run \cmd{godi\_console}

  \begin{enumerate}
    \item Select \cmd{conf-pcre} package (Packages are listed in
         alphabetical order.)

    Select [b]uild \& install, then e[x]it.

    GODI will give you the PCRE library configure options:

\begin{verbatim}
        [ 1] GODI_BASEPKG_PCRE = no
        [ 2]  GODI_PCRE_INCDIR =
        [ 3]  GODI_PCRE_LIBDIR =
\end{verbatim}

Select 2 to set the path to the directory containing pcre.h (version 5.0 or higher).

Select 3 to set the path to the directory containing libpcre.a (version 5.0 or higher).

Select e[x]it twice to return to the main menu.

\item (Optional) If you are planning to include support for Jungle,
Galax's secondary storage manager, then select \cmd{conf-bdb}.

     Select [b]uild \& install, then e[x]it.

     GODI will give you the BDB library configure options:

\begin{verbatim}
        [ 1] GODI_BDB_INCDIR =
        [ 2] GODI_BDB_LIBDIR =
\end{verbatim}

Select 1 to set the path to the directory containing db.h. (version 4.4 or higher).

Select 2 to set the path to the directory containing libdb.a (version 4.4 or higher).

Select e[x]it twice to return to the main menu.

     \item  Select \cmd{godi-galax} package.

     Select [b]uild \& install, then e[x]it.

     This will add all of Galax's dependencies to your GODI configuration. GODI will give you Galax's configure options:

\begin{verbatim}
		[ 1] GODI_GALAX_INCLUDE_JUNGLE = no
		[ 2]  GODI_GALAX_INCLUDE_C_API = no
		[ 3] GODI_GALAX_INCLUDE_JAVA_$ = no
		[ 4]   GODI_GALAX_INCLUDE_UTF8 = yes
		[ 5] GODI_GALAX_INCLUDE_ISO88$ = yes
		[ 6]      GODI_GALAX_JAVA_HOME = unset
		[ 7]       GODI_GALAX_JAVA_BIN = unset
		[ 8]   GODI_GALAX_JAVA_INCLUDE = unset
		[ 9] GODI_GALAX_REGRESSION_SU$ = unset
\end{verbatim}
%$

        You can change the default Galax configuration as follows:

        Select 1 to include Jungle, Galax's secondary storage manager.

        Select 2 to include Galax's C API. 

        Select 3 to include Galax's Java API (requires the C API).

        Select 4(5) to exclude UTF8(ISO88) character set (reduces sizes of executables).

        Select 6(7,8) to set the Java home(bin,include) directories.

        Select 9 to set directory where XQuery 1.0 test suite is installed.

        Select e[xi]t.

        Select [s]tart/continue, which will report all dependencies.

        Then select [s]tart/continue, which will report actual dependencies based on your current GODI installation.

         Then select [o]k.
  \end{enumerate}

\item Take a long coffee break...while GODI downloads and installs
      packages. 

\item Finish

Add \cmd{<godi-prefix-path>/bin} to your \cmd{PATH}.
        
When installation has completed, Galax will be installed in
\cmd{<godi-prefix-path>/} with the following subdirectories:

\begin{center}
{\small
\begin{tabular}{|l|l|l|}
  \hline
  Directory & Sub-directory & Content\\
  \hline
  \cmd{bin/}  && Galax's command-line executables\\
  \cmd{lib/}  && \\
  &  \cmd{pervasive.xq} & Signatures of XQuery 1.0 Function and Operators  \\
  &  \cmd{c/}           & C libraries and API header files\\
  & \cmd{java/}        & Java libraries and API interfaces\\
  &  \cmd{ocaml/pkg-lib/galax/}  & OCaml libraries and interfaces\\
  \cmd{share/galax/} && \\
  &   \cmd{examples/}&  Examples of using C, Java, and OCaml APIs\\
  &   \cmd{regress/}&   Test harness and configuration for the W3C XQuery Test Suite\\
  &   \cmd{usecases/}&  XQuery 1.0 usecases\\
  \hline
\end{tabular}
}
\end{center}

\end{enumerate}
\subsection{Installing Galax source by hand}

If you already have an OCaml installation, you can install the
Galax source by hand.  To do so, you need the following versions
of the OCaml compiler and libraries, and C libraries:

\begin{center}
\begin{tabular}{|l|l|l|}
\hline
Library & Version & GODI-Package\\
\hline
{OCaml compiler}\footnote{\ahrefurl{http://caml.inria.fr/ocaml/release.en.html}} & 3.12.1+rc1 & godi-ocaml\\
\hline
\multicolumn{3}{|c|}{OCaml Libraries}\\
\hline
findlib\footnote{\ahrefurl{http://www.ocaml-programming.de/packages/findlib-1.2.7.tar.gz}} & 1.2.7 & godi-findlib\\
Pcre-ocaml\footnote{\ahrefurl{http://www.ocaml.info/ocaml\_sources/pcre-ocaml-6.2.3.tar.gz}} & 6.2.3 & godi-pcre\\
Ocamlnet\footnote{\ahrefurl{http://www.ocaml-programming.de/packages/ocamlnet-3.4.1.tar.gz}} & 3.4.1 & godi-ocamlnet\\
Caml IDL\footnote{\ahrefurl{http://caml.inria.fr/pub/old\_caml\_site/distrib/bazar-ocaml/camlidl-1.05.tar.gz}} & 1.05 & godi-camlidl\\
PXP\footnote{\ahrefurl{http://www.ocaml-programming.de/packages/pxp-1.2.2.tar.gz}} & 1.2.2 & godi-pxp\\
Camomile\footnote{\ahrefurl{http://prdownloads.sourceforge.net/camomile/camomile-0.8.3.tar.bz2}} & 0.8.3 & godi-camomile\\
\hline
\multicolumn{3}{|c|}{C Libraries}\\
\hline
	Berkeley DB (libdb) &4.3 & Only for Jungle\\
	PCRE & \verb+>+4.5 & \\
\hline
\end{tabular}
\end{center}

\begin{enumerate}
\item Download Galax source from \ahref{http://galax.sourceforge.net/}.
\item Unzip and untar Galax in a directory of your choice.
%$
      \cmd{gunzip galax-\galaxversion.tar.gz}

      \cmd{tar xvf galax-\galaxversion.tar}

\item Configure Galax.

From the \verb+galax/+ you just created, run:

      \verb+./configure -galax-home $GALAXHOME+%$

      Where \verb+$GALAXHOME+ is the target installation directory.%$

      If you want the C API, add \cmd{-with-c}.

      If you want the Java API, add  \cmd{-with-java -java-home <Top-level Java directory>}.

      The configure script will try to find your OCaml installation
      and other necessary libraries by itself.  Review the default
      configuration before proceeding.  If you need to change the
      default configuration, run \cmd{./configure -help} for all
      options.

\item Build Galax.

      In \verb+galax/+ run: 
  
        \cmd{make world}

      Go get more coffee...

\item Install Galax.

      In \verb+galax/+ run: 
  
        \cmd{make install}

\item Finish

  Add \verb+$GALAXHOME/bin+ to your \cmd{PATH} environment variable.%$

      When installation has completed, Galax will be installed in
      \verb+$GALAXHOME+ with these subdirectories:%$

\begin{center}
\begin{tabular}{|l|l|}
\hline
Directory & Content\\
\hline
         \cmd{bin/}       & Command-line executables	 \\
         \cmd{examples/}  & Examples of using C, Java, and OCaml APIs \\
         \cmd{regress/}   & Test harness and configuration for the W3C XQuery Test Suite\\
         \cmd{usecases/}  & XQuery 1.0 usecases \\
\hline
\end{tabular}
\end{center}

\end{enumerate}
        
\eat{
\section{Operating-System Installation Notes}
\label{sec:os-notes}

\subsection{Linux}
\label{sec:os-notes-linux}

\subsubsection{ocamlnet}                                                                                                   If you have compiled OCaml with no shared libraries, you will get an
error when compiling ocamlnet, because 
the \cmd{src/netstring/tools/unimap\_to\_ocaml/unimap\_to\_ocaml} executable 
is a pre-compiled ocamlrun (bytecode) executable that assumes 
OCaml supports shared libraries.

Remove \cmd{src/netstring/tools/unimap\_to\_ocaml/unimap\_to\_ocaml}, remake in
that directory, then proceed with compilation and installation.

\subsection{Solaris}
\label{sec:os-notes-solaris}

\subsection{APIs}

The Solaris implementation requires gcc and the GNU binary utilities. 
If your gcc compiler is configured to invoke the native
binary utilities (e.g., as, ld, etc.) and not the GNU utilities, you
need to set up your environment so gcc will use the GNU utilities. 
Set \cmd{\$LD\_LIBRARY\_PATH} to gcc library directory and 
\cmd{\$GCC\_EXEC\_PREFIX} to the gcc binary utilities directory.

\subsection{MacOS}
\label{sec:os-notes-mac}

\subsubsection{OCaml:}  We recommend that when you configure the OCaml
compiler, use the \textbf{-no-shared-libs} options, which means that
the OCaml compiler will only create static libraries.

If you get linking errors when compiling the C API, you may have to
rebuild your OCaml compiler using the \textbf{-fno-common} compiler
option.  Edit your OCaml config/Makefile and add \textbf{-fno-common}
to 
\cmd{BYTECCCOMPOPTS} and \cmd{NATIVECCCOMPOPTS}.

\subsubsection{ocamlnet}                                                                                                                                          
If you have compiled OCaml with no shared libraries, you will get an
error when compiling ocamlnet, because 
the \cmd{src/netstring/tools/unimap\_to\_ocaml/unimap\_to\_ocaml} executable 
is a pre-compiled ocamlrun (bytecode) executable that assumes 
OCaml supports shared libraries.

Remove \cmd{src/netstring/tools/unimap\_to\_ocaml/unimap\_to\_ocaml}, 
remake in that directory, then proceed with ocamlnet compilation and installation.

If you get errors from ocamlfind, try:

\cmd{ocamlopt -o unimap\_to\_ocaml \$OCAMLHOME/str.cmxa unimap\_to\_ocaml.ml}

I had to generate unimap\_to\_ocaml by hand, because ocamlfind
did note find str.cmxa correctly. 

\subsubsection{pxp}

I was not able to build PXP successfully using the wlex and ulex
lexers and all the character encodings.  Instead, I configured PXP
with just two character encodings and the standard lexers:

\cmd{./configure -without-wlex -without-ulex -lexlist utf8,iso88591 -without-pp}

Then proceed with compilation and installation.

\subsection{Windows}
\label{sec:os-notes-mingw}
}

\section{XQuery Test Suite}
\label{sec:testsuite}

The Galax distribution includes a test harness for the W3C XQuery
tests suite. It can be run by following the steps listed below.

\begin{enumerate}
\item {\bf Download and install the XQuery Test Suite.} The latest
version of the XQuery Test Suite can be found on the W3C XQuery Web
page at the following address:

\ahrefurl{http://www.w3.org/XML/Query/test-suite/}

The Test Suite comes as a \verb+zip+ archive, which can be unzipped in
a directory of your choice (We use \verb+$XQTS+ in the following
instructions).
%$
\item {\bf Configure Galax's test harness.} If you followed the GODI
installation path, you need to select option 9 as indicated above to
specify the directory in which the test suite has been installed. If
you followed the ``by hand'' installation path, you must specify the
directory in which the test suite has been installed using the
following configure option:
\begin{verbatim}
-regression $XQTS
\end{verbatim}
%$
\item {\bf Run the test suite.} You can run the test suite by
calling:

\cmd{make}

from the \verb+$GALAXHOME/regress+ directory. This will produce a file
called:

\cmd{testresults-W3C.xml}

which contains the result of running the test suite (following the
format required for test results by the XQTS).
%$
\item {\bf Generate a test suite summary.} The XQTS distribution
contains an XSLT style sheet which can be used to produced a summary
of the test results in HTML form. The corresponding style sheet is
located in the directory:
\begin{verbatim}
$XQTS/ReportingResults
\end{verbatim}
%$
The stylesheet can be run by:
\begin{enumerate}
\item Specifying the path to the test results obtained from the
previous step in the configuration file:
\begin{verbatim}
$XQTS/ReportingResults/Results.xml
\end{verbatim}
%$
as follows:
\begin{alltt}
<results>
   <result>{\em\$GALAXHOME}/regress/testresults-W3C.xml</result>
</results>
\end{alltt}
%$
\item Changing to the directory:
\begin{verbatim}
$XQTS/ReportingResults/Results.xml
\end{verbatim}
%$
and running one of the two following commands (you will need a working
installation of ant, and of an XSLT processor):
\begin{verbatim}
ant -f Build.xml create
ant -f Build.xml createsimple
\end{verbatim}
which should respectively produce the following HTML files:
\begin{verbatim}
XQTSReportSimple.html
XQTSReport.html
\end{verbatim}
\end{enumerate}
\end{enumerate}

\section{Web Site Interface}
\label{sec:website}
  The Galax Web site and on-line demo are bundled with the source
  distribution (only).

\subsection{Prerequisites}

The Galax Web site has only been tested with Apache Web servers. We
recommend you use Apache as some of the CGI scripts might be sensitive
to the server you are using. Apache is commonly installed with most
Linux distributions, or can be downloaded from:
\ahrefurl{http://www.apache.org/}.

\subsection{Installation of Website Interface}
\begin{enumerate}
\item From the source directory, edit
\cmd{galax/website/Makefile.config} and initialize the following
variables:
\begin{description}
\item[\cmd{WEBSITE}]   Location of the Apache directory for Galax.
\item[\cmd{CGIBIN\_PREFIX}] Prefix of directory that may contain CGI
  executables.  If empty, then they are placed in \cmd{WEBSITE}. 
\end{description}
\item Compile the the demo scripts: \cmd{make}
\item Install the web site and demo scripts: \cmd{sudo make install}

\item Configure your own Apache server: If all the galax demo files
     (HTML, XML and CGIs) are placed in one directory where the HTTP
     daemon is expecting to find them, then it is necessary to add the
     following config info to the proper \cmd{http.conf} file:
\begin{verbatim}
  <Directory "/var/www/html/galax">
    Options All
    AllowOverride None
    AddHandler cgi-script .cgi 
    Order allow,deny
    Allow from all
  </Directory>
\end{verbatim}
     This permits scripts with suffix .cgi in
     \cmd{/var/www/html/galax} to be executed.  The Galax demo is
     available at \cmd{http://localhost/galax}.
\item OR Configure an existing multi-user Apache server.

     Your sysadmin may already have set up an Apache server for
     general use, and allows CGI programs by any user.  You can verify
     by finding directives similar to the following in httpd.conf
     (wherever it might be located on your system),

\begin{verbatim}
  AddHandler cgi-script .cgi
  <DirectoryMatch "/galax/cgi-bin">
    AllowOverride AuthConfig
    Options ExecCGI
    SetHandler cgi-script
  </DirectoryMatch>
\end{verbatim}

In that case, simply follow the comments in
\cmd{website/Makefile.config} to choose installation destinations for
your CGI programs and the HTML documents should suffice.  The URL for
accessing the installed site will depend on how your webserver is set
up.  Consult your sysadmin or webadmin for further help.
\end{enumerate}
