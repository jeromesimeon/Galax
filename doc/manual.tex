\documentclass{book}[11pt]

\usepackage{hevea}
\usepackage{fullpage}
\usepackage{alltt}
\usepackage[latin1]{inputenc}
\newcommand{\eat}[1]{}
\newcommand{\term}[1]{\textbf{#1}}
\newcommand{\func}[1]{\texttt{#1}}
\newcommand{\cmd}[1]{\textbf{\texttt{#1}}}
\newcommand{\code}[1]{\textbf{\texttt{#1}}}
\newcommand{\note}[1]{\textbf{Note:} #1}
\newcommand{\scream}[1]{\textbf{*** } #1 \textbf{***}}
\newcommand{\version}{1.0}
% Version of XQuery to which Galax is aligned 
\newcommand{\xqueryrec}{January, 2007} 
\newcommand{\xqueryversion}{\xqueryrec} 
\newcommand{\xquerytestsuiteurl}{\ahrefurl{http://www.w3.org/XML/Query/test-suite/}}
\newcommand{\xquerytestsuiteversion}{1.0.2}
\newcommand{\totaltests}{10,055}
\newcommand{\galaxtotaltests}{7917}
\newcommand{\galaxpassestests}{7852}
\newcommand{\xqueryurl}{\ahrefurl{http://www.w3c.org/TR/xquery}}
\newcommand{\xquerywgurl}{\ahrefurl{http://www.w3c.org/XML/Query}}
\newcommand{\galaxurl}{\ahrefurl{http://www.galaxquery.org}}
\newcommand{\galaxdistrib}{\galaxurl/distrib.html}
\newcommand{\datamodelurl}{\ahrefurl{http://www.w3c.org/TR/query-datamodel}}
\newcommand{\fandourl}{\ahrefurl{http://www.w3c.org/TR/xpath-functions}}
\newcommand{\fsurl}{\ahrefurl{http://www.w3c.org/TR/xquery-semantics}}
\newcommand{\usecasesurl}{\ahrefurl{http://www.w3c.org/TR/xquery-use-cases}}
\newcommand{\xmlurl}{\ahrefurl{http://www.w3.org/TR/2000/REC-xml-20001006}}
\newcommand{\xmlnsurl}{\ahrefurl{http://www.w3.org/TR/1999/REC-xml-names-19990114/}}
\newcommand{\xmlschemaoneurl}{\ahrefurl{http://www.w3.org/TR/2001/REC-xmlschema-1-20010502/}}
\newcommand{\xmlschematwourl}{\ahrefurl{http://www.w3.org/TR/2001/REC-xmlschema-2-20010502/}}
\newcommand{\xquerybangurl}{\ahrefurl{http://xquerybang.cs.washington.edu/}}
\newcommand{\godiurl}{\ahrefurl{http://godi.ocaml-programming.de/}}
\newcommand{\ultfurl}{\ahrefurl{http://www.w3.org/TR/xquery-update-10/}}
\newcommand{\ultfwd}{August, 2007}
\newcommand{\xqueryp}{\ahrefurl{http://www.ximep-2006.org/papers/Paper-Chamberlin-Carey.pdf}}
\newcommand{\galaxusers}{\mailto{galax-users@research.att.com}}
\newcommand{\galaxrequestusers}{\mailto{galax-users-request@research.att.com}}
\newcommand{\bugzillaurl}{\ahrefurl{http://bugzilla.galaxquery.net/}}

\htmlprefix{Galax Manual: }
\title{\Large\bf {\LARGE Galax Version
\version\ Documentation}\\[0.5cm] \textit{``The XQuery Implementation
for Discriminating Hackers''}}
\author{Mary Fern\'andez, AT\&T Labs Research \\ J\'er\^ome Sim\'eon, IBM T.J. Watson Research Center}
\date{\today}

\begin{document}
\maketitle
\tableofcontents

\part{User's Manual}
\chapter{Getting Started}
\label{sec:readme}
\cutname{readme.html}
\section{What is XQuery 1.0?}

XQuery 1.0 is a query language for XML, defined by the World-Wide Web
Consortium (W3C), under the XML activity.  XQuery is a powerful
language, which supports XPath 2.0 as a subset and includes
expressions to construct new XML documents, SQL-like expressions to
perform selection, joins, and sorting over collections of XML values,
operations on namespaces, and expressions over XML Schema types.
XQuery is a functional language, which comes with an extensive library
of built-in functions, and allows user to define their own functions.
More information about XQuery can be found of the XML Query Working
Group Web page\footnote{\xquerywgurl}.

\section{What is Galax?}

Galax is an implementation of XQuery 1.0 designed with the following
goals in mind: completeness, conformance, performance, and
extensibility. Galax is open-source, and has been used on a large
variety of real-life XML applications. Galax relies on a formally
specified and open architecture which is particularly well suited for
users interested in teaching XQuery, or in experimenting with
extensions of the language or optimizations.

Here is a list of the main Galax features.

\begin{itemize}
\item Galax implements the January 2007 (W3C Recommendation)
  set of specifications for XQuery.
\item Galax supports Minimal Conformance, as well as the following
  optional features: Static Typing, Full Axis, Module, and
  Serialization. The following table lists Galax 1.0 results with
  respect to the XQuery Tests Suite version 1.0.2:
\begin{center}
\begin{tabular}{|lrrrr|}
  \hline
  & Pass & Fail & Total & Percent\\\hline
  Minimal Conformance      &       14553 & 71 & 14637 & (99.4\%)\\\hline
  Optional Features &&&& \\
  \hspace*{0.5cm}Static Typing Feature   &      46    & 0   & 46 & (100\%)\\
  \hspace*{0.5cm}Full Axis Feature        &     130  & 0   & 130 & (100\%)\\
  \hspace*{0.5cm}Module Feature            &    32    & 0   & 32 & (100\%)\\\hline
\end{tabular}
\end{center}
\item Galax supports Unicode, with native support for the UTF-8 and
  ISO-8859-1 character encodings.
\item Galax is portable and runs on most modern platforms.
\item Galax includes a command-line interface, APIs for OCaml, C, and
  Java, and a simple Web-based interface.
\item Galax includes partial support for the Schema Import and Schema
  Validation optional features. Unimplemented features of XML Schema
  include simple types facets and derivation by extension and
  restriction on complex types.
\item Galax supports the XQuery 1.0 Update
Facility\footnote{\ultfurl}.
\item Galax supports the XQueryP scripting extension for
  XQuery\footnote{\xqueryp}.
\end{itemize}

\paragraph{Alpha features:}

The following features are experimental.

\begin{itemize}
\item Galax inludes support for SOAP and WSDL-based Web Services
  invocation;
\item Galax inludes support for Distributed XQuery development;
\item Galax's compiler includes an optimizer that supports state of
  the art database and programming language optimization, including:
  \begin{itemize}
  \item type-based optimizations, notably eliminating unnecessary type
    matching or casting operations and turning dynamic dispatch into
    static dispatch for comparison and arithmetic operators;
  \item tail-recursion optimization;
  \item join and query unnesting optimizations, including physical
    support for XQuery-specific hash and sort join algorithms;
  \item tree-pattern detection in query plans and physical support for
    the TwigJoin and StaircaseJoin algorithms;
  \item generation of hybrid streamed/materialized query plans.
  \end{itemize}
\item Galax has an \emph{extensible data model} interface. This allows
  users to provide their own implementation of the XML data
  model. This can be used to support e.g., XML queries over legacy
  data.
\end{itemize}

\paragraph{Limitations:} See Chapter~\ref{sec:alignment} for details
  on Galax's alignment with the XQuery and XPath working drafts.

\subsection{Changes since the last version}

The following lists the main changes included with this version
(\version).

\begin{itemize}
\item Galax 1.0 supports the latest OCaml 3.10 compiler and GODI
3.10.
\item Support for the XQuery Update Facility 1.0, August 2007 Working
Draft. Support for the XQuery 1.0 Update Facility Static Typing
Feature.
\item Improved implementation of XQueryP.
\begin{itemize}
\item Re-implementation of while loops.
\item Fixed issues with scoping.
\end{itemize}
\item Improved support for modules.
  \begin{itemize}
  \item Properly implemented nested imports.
  \item Added support for module interfaces.
  \end{itemize}
\item Support for XQueryX trivial embedding.
\item Bug fixes:
  \begin{itemize}
  \item Fixed problems with support for in-scope namespaces.
  \item Fixed problems with checking of cyclic variable declarations.
  \item Fixed several problems with the parser, using wrong lexical states inside constructors.
  \item Small bug fixes in support for the prolog.
  \end{itemize}
\end{itemize}

Changes from older versions can be found in
Chapter~\ref{sec:releasenotes}.

\section{Downloading and installing Galax}

The official distribution can be downloaded from the main Galax Web
site\footnote{\galaxurl}.  Detailed installation instructions are
provided in Chapter~\ref{sec:install}.

\section{How to use Galax}

The Galax processor offers the following user interfaces:

\begin{itemize}
\item Command-line.
\item Application-programming interfaces (APIs) for OCaml, C, or Java.
\item Web interface.
\end{itemize}

\subsection{Using Galax from the command line}

A number of stand-alone command-line tools are provided with the Galax
distribution. Assuming the Galax distribution is intalled in
\verb+$GALAXHOME+, and that Galax executables are reachable from your
\verb+$PATH+ environment variable, The following examples show how to
use the main command-line tools.

\begin{description}
\item[glx xquery] The main XQuery interpreter (\term{glx xquery}) is
  the simplest way to use Galax. For instance, the following commands:
\begin{verbatim}
% echo "<two>{ 1+1 }</two>" > test.xq
% glx xquery test.xq
<two>2</two>
\end{verbatim}
evaluates the query \verb|<two>{ 1+1 }</two>| and prints the XML
result \verb|<two>2</two>|.

\item[glx xml] The Galax XML parser and XML Schema validator can be
  called as a standalone tool. For instance, the following command
  validates the document in \texttt{hispo.xml} against the schema in
  \texttt{hispo.xsd}:
\begin{verbatim}
% glx xml -validate -xmlschema $GALAXHOME/examples/docs/hispo.xsd \
                               $GALAXHOME/examples/docs/hispo.xml 
\end{verbatim}

\item[glx xmlschema] A stand-alone tool that maps XML Schema documents into the XQuery
  type system.   This tools is useful for checking whether Galax
  recognizes all the constructs in your XML Schema.  It also eliminates a lot of the
  ``noise'' in XML Schema's XML syntax. 

For instance, this command will print out the XQuery type
representation of the schema in \texttt{hispo.xsd}: 
\begin{verbatim}
% glx xmlschema $GALAXHOME/examples/docs/hispo.xsd 
\end{verbatim}
\end{description}

Chapter~\ref{sec:commandline} describes the command-line tools in detail.

\subsection{Web interface}

The Web interface is a simple and convenient way to get acquainted
with Galax. It allows users to submit a query, and view the result of
compilation and execution for that query.

An on-line version is available on-line at:
\ahrefurl{http://www.galaxquery.org/demo/galax\_demo.html}

You can also re-compile the demo from the Galax source and install it
on your own system. You will need an HTTP server (Apache is
recommended), and follow the compilation instructions in
Section~\ref{sec:website}.

\subsection{Language API's}

Galax supports APIs for OCaml, C, and Java.  See Chapter~\ref{sec:api}
for how to use the APIs.
  
If you have installed the binary distribution of Galax, all three APIs
are available.

If you have intalled the source distribution of Galax, you will need
to select the language(s) for which you need API support at
configuration time.  See Chapter~\ref{sec:install} for details on
compiling Galax from source.

Examples of how to use Galax's APIs can be found in the following
directories:\\ \cmd{\$GALAXHOME/examples/caml\_api/}\\
\cmd{\$GALAXHOME/examples/c\_api/}\\
\cmd{\$GALAXHOME/examples/java\_api/}.



\chapter{Installation}
\label{sec:install}
\cutname{install.html}
This chapter contains the following sections:
\begin{itemize}
\item Prerequisites (Section~\ref{sec:prelim}) : prerequisites for
  Galax installation.
\item Source Distribution (Section~\ref{sec:src-dist}) :  
  Installation instructions for the source distribution.
\eat{
\item Binary Distribution (Section~\ref{sec:bin-dist}) : requirements and
  installation instructions for the binary distribution.
\item Operating-System Installation Notes (Section~\ref{sec:os-notes})
  : detailed  installation notes for each supported target.
}
\item Running the W3C XQuery Test Suite (XQTS)
(Section~\ref{sec:testsuite}).
\item Web-form interface (Section~\ref{sec:website}) : installation
instructions for Galax's web-form interface.
\end{itemize}

\section{Prerequisites}
\label{sec:prelim}
\begin{itemize}
\item PCRE 5.0 or higher is required. If you do not have PCRE version
  5.0 or higher\footnote{\ahrefurl{http://www.pcre.org/}} installed.
\item (Optional) If you plan to use Jungle, Galax's secondary storage
  manager, you will need Berkeley DB version 4.4 or
  higher\footnote{\ahrefurl{http://dev.sleepycat.com/downloads/releasehistorybdb.html}}
  installed.
\end{itemize}

\section{Source Installation}
\label{sec:src-dist}

There are two methods for installing Galax from source:

\begin{enumerate}
\item \textbf{RECOMMENDED METHOD:} Use GODI installation and configuration tool
     to automatically install Galax and all the libraries on which it
     depends.

\item Install Galax and the libraries on which it depends manually.

This method may be preferable if you are an OCaml user, already have
an installation of OCaml and its libraries, and do not want to install
a new version of OCaml.  Even if you are already an OCaml user, the
first method is always recommended.
\end{enumerate}

\subsection{
Installing Galax using GODI installation and configuration tool}

\textbf{Warning:} Installing packages using GODI is an ON-LINE process, so
preferably, you should be connected to the Net by a fast, hard-wire
connection.

\begin{enumerate}
\item Download GODI from : \ahrefurl{http://godi.ocaml-programming.de/}

Follow GODI installation instructions through "bootstrap\_stage2"
command, adding the\\\verb+<godi-prefix-path>/bin+ and
\verb+<godi-prefix-path>/sbin+ directories to your PATH as instructed.

\textbf{Note:} For OCaml users, after GODI bootstrapping phase, the
OCaml executables are located in \verb+<godi-prefix-path>/bin+ and the
OCaml libraries are located in \verb+<godi-prefix-path>/lib/ocaml+.

\item Run \cmd{godi\_console}

  \begin{enumerate}
    \item Select \cmd{conf-pcre} package (Packages are listed in
         alphabetical order.)

    Select [b]uild \& install, then e[x]it.

    GODI will give you the PCRE library configure options:

\begin{verbatim}
        [ 1] GODI_BASEPKG_PCRE = no
        [ 2]  GODI_PCRE_INCDIR =
        [ 3]  GODI_PCRE_LIBDIR =
\end{verbatim}

Select 2 to set the path to the directory containing pcre.h (version 5.0 or higher).

Select 3 to set the path to the directory containing libpcre.a (version 5.0 or higher).

Select e[x]it twice to return to the main menu.

\item (Optional) If you are planning to include support for Jungle,
Galax's secondary storage manager, then select \cmd{conf-bdb}.

     Select [b]uild \& install, then e[x]it.

     GODI will give you the BDB library configure options:

\begin{verbatim}
        [ 1] GODI_BDB_INCDIR =
        [ 2] GODI_BDB_LIBDIR =
\end{verbatim}

Select 1 to set the path to the directory containing db.h. (version 4.4 or higher).

Select 2 to set the path to the directory containing libdb.a (version 4.4 or higher).

Select e[x]it twice to return to the main menu.

     \item  Select \cmd{godi-galax} package.

     Select [b]uild \& install, then e[x]it.

     This will add all of Galax's dependencies to your GODI configuration. GODI will give you Galax's configure options:

\begin{verbatim}
		[ 1] GODI_GALAX_INCLUDE_JUNGLE = no
		[ 2]  GODI_GALAX_INCLUDE_C_API = no
		[ 3] GODI_GALAX_INCLUDE_JAVA_$ = no
		[ 4]   GODI_GALAX_INCLUDE_UTF8 = yes
		[ 5] GODI_GALAX_INCLUDE_ISO88$ = yes
		[ 6]      GODI_GALAX_JAVA_HOME = unset
		[ 7]       GODI_GALAX_JAVA_BIN = unset
		[ 8]   GODI_GALAX_JAVA_INCLUDE = unset
		[ 9] GODI_GALAX_REGRESSION_SU$ = unset
\end{verbatim}
%$

        You can change the default Galax configuration as follows:

        Select 1 to include Jungle, Galax's secondary storage manager.

        Select 2 to include Galax's C API. 

        Select 3 to include Galax's Java API (requires the C API).

        Select 4(5) to exclude UTF8(ISO88) character set (reduces sizes of executables).

        Select 6(7,8) to set the Java home(bin,include) directories.

        Select 9 to set directory where XQuery 1.0 test suite is installed.

        Select e[xi]t.

        Select [s]tart/continue, which will report all dependencies.

        Then select [s]tart/continue, which will report actual dependencies based on your current GODI installation.

         Then select [o]k.
  \end{enumerate}

\item Take a long coffee break...while GODI downloads and installs
      packages. 

\item Finish

Add \cmd{<godi-prefix-path>/bin} to your \cmd{PATH}.
        
When installation has completed, Galax will be installed in
\cmd{<godi-prefix-path>/} with the following subdirectories:

\begin{center}
{\small
\begin{tabular}{|l|l|l|}
  \hline
  Directory & Sub-directory & Content\\
  \hline
  \cmd{bin/}  && Galax's command-line executables\\
  \cmd{lib/}  && \\
  &  \cmd{pervasive.xq} & Signatures of XQuery 1.0 Function and Operators  \\
  &  \cmd{c/}           & C libraries and API header files\\
  & \cmd{java/}        & Java libraries and API interfaces\\
  &  \cmd{ocaml/pkg-lib/galax/}  & OCaml libraries and interfaces\\
  \cmd{share/galax/} && \\
  &   \cmd{examples/}&  Examples of using C, Java, and OCaml APIs\\
  &   \cmd{regress/}&   Test harness and configuration for the W3C XQuery Test Suite\\
  &   \cmd{usecases/}&  XQuery 1.0 usecases\\
  \hline
\end{tabular}
}
\end{center}

\end{enumerate}
\subsection{Installing Galax source by hand}

If you already have an OCaml installation, you can install the
Galax source by hand.  To do so, you need the following versions
of the OCaml compiler and libraries, and C libraries:

\begin{center}
\begin{tabular}{|l|l|l|}
\hline
Library & Version & GODI-Package\\
\hline
{OCaml compiler}\footnote{\ahrefurl{http://caml.inria.fr/ocaml/release.en.html}} & 3.10 & godi-ocaml\\
\hline
\multicolumn{3}{|c|}{OCaml Libraries}\\
\hline
findlib\footnote{\ahrefurl{http://www.ocaml-programming.de/packages/findlib-1.1.2pl1.tar.gz}} & 1.1.2pl1 & godi-findlib\\
Pcre-ocaml\footnote{\ahrefurl{http://www.ocaml.info/ocaml\_sources/pcre-ocaml-5.12.2.tar.gz}} & 5.12.2 & godi-pcre\\
Ocamlnet\footnote{\ahrefurl{http://www.ocaml-programming.de/packages/ocamlnet-2.2.8.1.tar.gz}} & 2.2.8.1 & godi-ocamlnet\\
Caml IDL\footnote{\ahrefurl{http://caml.inria.fr/pub/old\_caml\_site/distrib/bazar-ocaml/camlidl-1.05.tar.gz}} & 1.05 & godi-camlidl\\
PXP\footnote{\ahrefurl{http://www.ocaml-programming.de/packages/pxp-1.2.0test1.tar.gz}} & 1.2.0test1 & godi-pxp\\
Camomile\footnote{\ahrefurl{http://prdownloads.sourceforge.net/camomile/camomile-0.7.1.tar.bz2}} & 0.7.1 & godi-camomile\\
\hline
\multicolumn{3}{|c|}{C Libraries}\\
\hline
	Berkeley DB (libdb) &4.3 & Only for Jungle\\
	PCRE & \verb+>+4.5 & \\
\hline
\end{tabular}
\end{center}

\begin{enumerate}
\item Download Galax source from \ahref{http://galax.sourceforge.net/}.
\item Unzip and untar Galax in a directory of your choice.
%$
      \cmd{gunzip galax-\version.tar.gz}

      \cmd{tar xvf galax-\version.tar}

\item Configure Galax.

From the \verb+galax/+ you just created, run:

      \verb+./configure -galax-home $GALAXHOME+%$

      Where \verb+$GALAXHOME+ is the target installation directory.%$

      If you want the C API, add \cmd{-with-c}.

      If you want the Java API, add  \cmd{-with-java -java-home <Top-level Java directory>}.

      The configure script will try to find your OCaml installation
      and other necessary libraries by itself.  Review the default
      configuration before proceeding.  If you need to change the
      default configuration, run \cmd{./configure -help} for all
      options.

\item Build Galax.

      In \verb+galax/+ run: 
  
        \cmd{make world}

      Go get more coffee...

\item Install Galax.

      In \verb+galax/+ run: 
  
        \cmd{make install}

\item Finish

  Add \verb+$GALAXHOME/bin+ to your \cmd{PATH} environment variable.%$

      When installation has completed, Galax will be installed in
      \verb+$GALAXHOME+ with these subdirectories:%$

\begin{center}
\begin{tabular}{|l|l|}
\hline
Directory & Content\\
\hline
         \cmd{bin/}       & Command-line executables	 \\
         \cmd{examples/}  & Examples of using C, Java, and OCaml APIs \\
         \cmd{regress/}   & Test harness and configuration for the W3C XQuery Test Suite\\
         \cmd{usecases/}  & XQuery 1.0 usecases \\
\hline
\end{tabular}
\end{center}

\end{enumerate}
        
\eat{
\section{Operating-System Installation Notes}
\label{sec:os-notes}

\subsection{Linux}
\label{sec:os-notes-linux}

\subsubsection{ocamlnet}                                                                                                   If you have compiled OCaml with no shared libraries, you will get an
error when compiling ocamlnet, because 
the \cmd{src/netstring/tools/unimap\_to\_ocaml/unimap\_to\_ocaml} executable 
is a pre-compiled ocamlrun (bytecode) executable that assumes 
OCaml supports shared libraries.

Remove \cmd{src/netstring/tools/unimap\_to\_ocaml/unimap\_to\_ocaml}, remake in
that directory, then proceed with compilation and installation.

\subsection{Solaris}
\label{sec:os-notes-solaris}

\subsection{APIs}

The Solaris implementation requires gcc and the GNU binary utilities. 
If your gcc compiler is configured to invoke the native
binary utilities (e.g., as, ld, etc.) and not the GNU utilities, you
need to set up your environment so gcc will use the GNU utilities. 
Set \cmd{\$LD\_LIBRARY\_PATH} to gcc library directory and 
\cmd{\$GCC\_EXEC\_PREFIX} to the gcc binary utilities directory.

\subsection{MacOS}
\label{sec:os-notes-mac}

\subsubsection{OCaml:}  We recommend that when you configure the OCaml
compiler, use the \textbf{-no-shared-libs} options, which means that
the OCaml compiler will only create static libraries.

If you get linking errors when compiling the C API, you may have to
rebuild your OCaml compiler using the \textbf{-fno-common} compiler
option.  Edit your OCaml config/Makefile and add \textbf{-fno-common}
to 
\cmd{BYTECCCOMPOPTS} and \cmd{NATIVECCCOMPOPTS}.

\subsubsection{ocamlnet}                                                                                                                                          
If you have compiled OCaml with no shared libraries, you will get an
error when compiling ocamlnet, because 
the \cmd{src/netstring/tools/unimap\_to\_ocaml/unimap\_to\_ocaml} executable 
is a pre-compiled ocamlrun (bytecode) executable that assumes 
OCaml supports shared libraries.

Remove \cmd{src/netstring/tools/unimap\_to\_ocaml/unimap\_to\_ocaml}, 
remake in that directory, then proceed with ocamlnet compilation and installation.

If you get errors from ocamlfind, try:

\cmd{ocamlopt -o unimap\_to\_ocaml \$OCAMLHOME/str.cmxa unimap\_to\_ocaml.ml}

I had to generate unimap\_to\_ocaml by hand, because ocamlfind
did note find str.cmxa correctly. 

\subsubsection{pxp}

I was not able to build PXP successfully using the wlex and ulex
lexers and all the character encodings.  Instead, I configured PXP
with just two character encodings and the standard lexers:

\cmd{./configure -without-wlex -without-ulex -lexlist utf8,iso88591 -without-pp}

Then proceed with compilation and installation.

\subsection{Windows}
\label{sec:os-notes-mingw}
}

\section{XQuery Test Suite}
\label{sec:testsuite}

The Galax distribution includes a test harness for the W3C XQuery
tests suite. It can be run by following the steps listed below.

\begin{enumerate}
\item {\bf Download and install the XQuery Test Suite.} The latest
version of the XQuery Test Suite can be found on the W3C XQuery Web
page at the following address:

\ahrefurl{http://www.w3.org/XML/Query/test-suite/}

The Test Suite comes as a \verb+zip+ archive, which can be unzipped in
a directory of your choice (We use \verb+$XQTS+ in the following
instructions).
%$
\item {\bf Configure Galax's test harness.} If you followed the GODI
installation path, you need to select option 9 as indicated above to
specify the directory in which the test suite has been installed. If
you followed the ``by hand'' installation path, you must specify the
directory in which the test suite has been installed using the
following configure option:
\begin{verbatim}
-regression $XQTS
\end{verbatim}
%$
\item {\bf Run the test suite.} You can run the test suite by
calling:

\cmd{make}

from the \verb+$GALAXHOME/regress+ directory. This will produce a file
called:

\cmd{testresults-W3C.xml}

which contains the result of running the test suite (following the
format required for test results by the XQTS).
%$
\item {\bf Generate a test suite summary.} The XQTS distribution
contains an XSLT style sheet which can be used to produced a summary
of the test results in HTML form. The corresponding style sheet is
located in the directory:
\begin{verbatim}
$XQTS/ReportingResults
\end{verbatim}
%$
The stylesheet can be run by:
\begin{enumerate}
\item Specifying the path to the test results obtained from the
previous step in the configuration file:
\begin{verbatim}
$XQTS/ReportingResults/Results.xml
\end{verbatim}
%$
as follows:
\begin{alltt}
<results>
   <result>{\em\$GALAXHOME}/regress/testresults-W3C.xml</result>
</results>
\end{alltt}
%$
\item Changing to the directory:
\begin{verbatim}
$XQTS/ReportingResults/Results.xml
\end{verbatim}
%$
and running one of the two following commands (you will need a working
installation of ant, and of an XSLT processor):
\begin{verbatim}
ant -f Build.xml create
ant -f Build.xml createsimple
\end{verbatim}
which should respectively produce the following HTML files:
\begin{verbatim}
XQTSReportSimple.html
XQTSReport.html
\end{verbatim}
\end{enumerate}
\end{enumerate}

\section{Web Site Interface}
\label{sec:website}
  The Galax Web site and on-line demo are bundled with the source
  distribution (only).

\subsection{Prerequisites}

The Galax Web site has only been tested with Apache Web servers. We
recommend you use Apache as some of the CGI scripts might be sensitive
to the server you are using. Apache is commonly installed with most
Linux distributions, or can be downloaded from:
\ahrefurl{http://www.apache.org/}.

\subsection{Installation of Website Interface}
\begin{enumerate}
\item From the source directory, edit
\cmd{galax/website/Makefile.config} and initialize the following
variables:
\begin{description}
\item[\cmd{WEBSITE}]   Location of the Apache directory for Galax.
\item[\cmd{CGIBIN\_PREFIX}] Prefix of directory that may contain CGI
  executables.  If empty, then they are placed in \cmd{WEBSITE}. 
\end{description}
\item Compile the the demo scripts: \cmd{make}
\item Install the web site and demo scripts: \cmd{sudo make install}

\item Configure your own Apache server: If all the galax demo files
     (HTML, XML and CGIs) are placed in one directory where the HTTP
     daemon is expecting to find them, then it is necessary to add the
     following config info to the proper \cmd{http.conf} file:
\begin{verbatim}
  <Directory "/var/www/html/galax">
    Options All
    AllowOverride None
    AddHandler cgi-script .cgi 
    Order allow,deny
    Allow from all
  </Directory>
\end{verbatim}
     This permits scripts with suffix .cgi in
     \cmd{/var/www/html/galax} to be executed.  The Galax demo is
     available at \cmd{http://localhost/galax}.
\item OR Configure an existing multi-user Apache server.

     Your sysadmin may already have set up an Apache server for
     general use, and allows CGI programs by any user.  You can verify
     by finding directives similar to the following in httpd.conf
     (wherever it might be located on your system),

\begin{verbatim}
  AddHandler cgi-script .cgi
  <DirectoryMatch "/galax/cgi-bin">
    AllowOverride AuthConfig
    Options ExecCGI
    SetHandler cgi-script
  </DirectoryMatch>
\end{verbatim}

In that case, simply follow the comments in
\cmd{website/Makefile.config} to choose installation destinations for
your CGI programs and the HTML documents should suffice.  The URL for
accessing the installed site will depend on how your webserver is set
up.  Consult your sysadmin or webadmin for further help.
\end{enumerate}


\chapter{Tutorial}
\label{sec:tutorial}
\cutname{tutorial.html}
\section{Executing a query}

The simplest way to use Galax is by calling the \term{glx xquery}
interpreter from the command line.  This chapter describes the most
frequently used command-line options.  Chapter~\ref{sec:commandline}
enumerates all the command-line options.

Before you begin, follow the instructions in
Section~\ref{sec:src-dist} and run the following query to make sure
your environment is set-up correctly:
\begin{verbatim}
% echo '<two>{ 1+1 }</two>' > test.xq
% glx xquery test.xq
<two>2</two>
\end{verbatim}

Galax evaluates expression \verb|<two>{ 1+1 }</two>| in file
\code{test.xq} and prints the result \verb|<two>2</two>|.

By default, Galax parses and evaluates an XQuery main module, which
contains both a prolog and an expression.  Sometimes it is useful to
separate the prolog from an expression, for example, if the same
prolog is used by multiple expressions.   The \code{-context} option
specifies a file that contains a query prolog.  

All of the XQuery use cases in \cmd{\$GALAXHOME/usecases} are
implemented by separating the query prolog from the query expressions.
Here is how to execute the Parts usecase:
\begin{alltt}
% cd $GALAXHOME/usecases 
% glx xquery -context parts_context.xq parts_usecase.xq
\end{alltt}

The other use cases are executed similarly, for example:
\begin{alltt}
% glx xquery -context rel_context.xq rel_usecase.xq
\end{alltt}

\section{Accessing Input}

You can access an input document by calling the
\code{fn:doc()} function and passing the file name as an argument:
\begin{alltt}
% cd $GALAXHOME/usecases 
% echo 'fn:doc("docs/books.xml")' > doc.xq
% glx xquery doc.xq
\end{alltt}

You can access an input document by referring to the context item (the
``.'' dot variable), whose value is the document's content:
\begin{alltt}
% echo '.' >dot.xq
% glx xquery -context-item docs/books.xml dot.xq
\end{alltt}

You can also access an input document by using the \cmd{-doc}
argument, which binds an external variable to the content of the given
document file:
\begin{alltt}
% echo 'declare variable $x external; $x' > var.xq
% glx xquery -doc x=docs/books.xml var.xq
\end{alltt}

\section{Controlling Output}

By default, Galax serializes the result of a query in a format that
reflects the precise data model instance. For example, the result of
this query is serialized as the literal \code{2}:
\begin{alltt}
% echo "document \{ 1+1 \}"> docnode.xq
% glx xquery docnode.xq
document \{ 2 \}
\end{alltt}

If you want the output of your query to be as the standard prescribes, 
then use the \code{-serialize standard} option:
\begin{alltt}
% glx xquery docnode.xq -serialize standard
2
\end{alltt}

\eat{
If you want the output of your query to be a well-formed XML value,
then use the \code{-serialize wf} option:
\begin{alltt}
% glx xquery docnode.xq -serialize wf
\end{alltt}
The result of this query is:
\begin{alltt}
<?xml version="1.0" encoding="UTF-8"?>
<glx:result xmlns:glx="http://www.galaxquery.org">2</glx:result>
\end{alltt}
}

By default, Galax serializes the result value to standard output.  Use
the \cmd{-output-xml} option to serialize the result value to an output
file.
\begin{alltt}
% glx xquery docnode.xq -serialize standard -output-xml output.xml
% cat output.xml
2
\end{alltt}

\section{Controlling Compilation}

By default, Galax compiles the given query an returns the
corresponding result. The following options can be set to print the
query as it progresses through the compilation pipeline .

\begin{alltt}
  -print-expr [on/off] \emph{Print input expression}
  -print-normalized-expr [on/off] \emph{Print expression after normalization}
  -print-rewritten-expr [on/off] \emph{Print expression after rewriting}
  -print-logical-plan [on/off] \emph{Print logical plan}
  -print-optimized-plan [on/off] \emph{Print logical plan after optimization}
  -print-physical-plan [on/off] \emph{Print physical plan}
\end{alltt}

As the output for the compiled query can be quite large, it is often
convenient to set the output to verbose using \code{-verbose on},
which prints headers for each phase. For instance, the following
command prints the original query, and the optimized logical plan for
the query.

\begin{alltt}
% glx xquery docnode.xq -verbose on -print-expr on -print-optimized-plan on
\end{alltt}

\section{Updates and Procedural Extensions}

Galax supports several extensions to XQuery 1.0, notably XML updates
and a procedural extensions. To enable one of those extensions, you
must use the corresponding language level option on the command line:

\begin{alltt}
glx xquery -language ultf       (: W3C Update Facility :)
glx xquery -language xquerybang (: XQuery! Language :)
glx xquery -language xqueryp    (: XQueryP Language :)
\end{alltt}

Some examples of each of the three languages are provided in the
\verb+$GALAXHOME/examples/extensions+ directory.


\part{Reference Manual}

\chapter{XQuery 1.0 Conformance}
\label{sec:alignment}
\cutname{alignment.html}
This chapter documents the relationship of Galax to the target W3C
working drafts. Galax \galaxversion\ is a prototype implementation,
and therefore it is not (yet) completely aligned with the relevant W3C
working drafts (WDs).  This chapter also document the non-standard
features in Galax \galaxversion\ and the known bugs and limitations.

Galax \galaxversion\ implements the \xqueryversion\ XQuery 1.0 and
XPath 2.0 Recommendations (Second Edition), the XML 1.0
Recommendation, the Namespaces in XML Recommendation, and XML Schema
Recommendation (Parts 1 and 2).

Galax \galaxversion\ implements the XQuery 1.0 Specifications:
\begin{itemize}
\item XQuery 1.0 and XPath 2.0 Data Model.
      W3C Recommendation \xqueryrec.
      (\ahrefurl{http://www.w3.org/TR/xpath-datamodel/}).
\item XQuery 1.0 : An XML Query Language.
      W3C Recommendation \xqueryrec.
      (\ahrefurl{http://www.w3.org/TR/xquery/}).
\item XQuery 1.0 and XPath 2.0 Formal Semantics.
      W3C Recommendation \xqueryrec.
      (\ahrefurl{http://www.w3.org/TR/xquery-semantics/}).
\item XQuery 1.0 and XPath 2.0 Functions and Operators.
      W3C Recommendation \xqueryrec.
      (\ahrefurl{http://www.w3.org/TR/xpath-functions/}).
\item XML Query Use Cases.
      W3C Working Draft 23 March 2007.
      (\ahrefurl{http://www.w3.org/TR/2007/NOTE-xquery-use-cases-20070323/}).
\item XML Schema Part 1: Structures and Part 2: Datatypes.
      W3C Recommendation 2 May 2001.
      \ahrefurl{http://www.w3.org/TR/xmlschema-1/},
      \ahrefurl{http://www.w3.org/TR/xmlschema-2/}.
\end{itemize}

\section{Data Model}

Galax \galaxversion\ fully supports the XQuery 1.0 and XPath 2.0 Data
Model.

Galax \galaxversion\ implements an xsd:float value as an xsd:double
value.

\section{XQuery}

The alignment issues in this section follow the outline of the
"Expressions" section in \ahrefurl{http://www.w3c.org/TR/xquery}.  If
a subsection is not listed here, it means that Galax \galaxversion\
implements the semantics described in that section.

\subsection*{\ahref{\rawxqueryurl\#static_context}{XQuery Section 2.1.1 Static Context}}

   Galax \galaxversion\ does not support:
\begin{itemize}
\item \term{XPath 1.0 compatibility modes}.
\item \term{Collations}.
\item The \term{construction mode}.  Type annotations on copied
  elements are always erased/eliminated. 
\item The \term{ordering mode}.  Input order is always preserved. 
\end{itemize}

\subsection*{\ahref{\rawxqueryurl\#dynamic_context}{XQuery Section 2.1.2 Dynamic Context}}
The \term{implicit timezone} is set to the local timezone. 

\subsection*{\ahref{\rawxqueryurl\#id-processing-model}{XQuery Section 2.2 Processing Model}}

Galax's processing model is similar to XQuery's abstract processing
model.  See Section~\ref{sec:developers} for more information on
Galax's internal processing model.

\subsection*{\ahref{\rawxqueryurl\#id-serialization}{XQuery Section 2.2.4 Serialization}}

\subsection*{\ahref{\rawxqueryurl\#id-ebv}{XQuery Section 2.3.3 Effective Boolean Value}}

   Galax \galaxversion\ does not check that a numeric value is equal
   to NaN when computing an effective boolean value.

\subsection*{\ahref{\rawxqueryurl\#id-input-sources}{XQuery Section 2.3.4 Input Sources}}

   Galax \galaxversion\ does not support the \func{fn:collection()}
   function.
 
   The context item and values for external variables can be specified
   on the command line or in the API.  See
   Sections~\ref{sec:inputoptions} and~\ref{sec:quickstart}.

\subsection*{\ahref{\rawxqueryurl\#id-sequencetype-matching}{XQuery Section 2.4.4 SequenceType Matching}}

Galax requires that all actual types, that is, those types that annotate
input documents be in the in-scope schema definitions.   Galax
will raise a dynamic error if it encounters a type in a document that
is not imported into the query by an \term{import schema} prolog statement.

\subsection*{\ahref{\rawxqueryurl\#id-optional-features}{XQuery Section 2.6 Optional Features}}

Galax supports the \term{Schema Import}, \term{Static Typing}, and
\term{Full Axis} features. 

\subsection*{\ahref{\rawxqueryurl\#id-module-feature}{XQuery Section 2.6.4 Module Feature}}

   Galax \galaxversion\ supports the Module feature.

\subsection*{\ahref{\rawxqueryurl\#id-pragmas}{XQuery Section 2.6.5 Pragmas}}

   Galax \galaxversion\ does not support the \term{Pragmas} feature.

\subsection*{\ahref{\rawxqueryurl\#id-must-understand}{XQuery Section 2.6.6 Must-Understand Extensions}}

   Galax \galaxversion\ does not support must-understand extensions.

\subsection*{\ahref{\rawxqueryurl\#id-static-extensions}{XQuery Section 2.6.7 Static Typing Extensions}}

   Galax \galaxversion\ does not support static typing extensions.

\subsection*{\ahref{\rawxqueryurl\#id-literals}{XQuery Section 3.1.1 Literals}}

   Galax \galaxversion\ implements an xsd:float value as an xsd:double value.

\subsection*{\ahref{\rawxqueryurl\#axes}{XQuery Section 3.2.1.1 Axes}}

   Galax \galaxversion\ supports all axes with the exception of the
   \term{preceding} and \term{following} axes.

\subsection*{\ahref{\rawxqueryurl\#id-constructors}{XQuery Section 3.7 Constructors}}

When constructing a new element, Galax \galaxversion\ always
erases/eliminates type annotations on copied elements.

When constructing a new element, Galax \galaxversion\ requires that 
the new element's attributes precede its other content. 

\subsection*{\ahref{\rawxqueryurl\#id-ns-nodes-on-elements}{XQuery Section 3.7.4 In-scope Namespaces of a Constructed Element}}

Namespace declarations in input and output documents and in input
queries are not handled consistently. 
We are working on
this.

\subsection*{\ahref{\rawxqueryurl\#id-unordered-expressions}{XQuery 
Section 3.9 Ordered and Unordered Expressions}}

Galax \galaxversion\ will accept queries that contain
   the \code{ordered} or \code{unordered} expressions, but they have
   no effect on query evaluation (i.e., they are no-ops).

\subsection*{\ahref{\rawxqueryurl\#id-module-declaration}{XQuery Section 4.2 Module Declaration}}
\subsection*{\ahref{\rawxqueryurl\#id-module-imports}{XQuery Section 4.2 Module Import}}

Galax \galaxversion\ supports the Module feature.

\subsection*{\ahref{\rawxqueryurl\#id-default-collation-declaration}{XQuery
    Section 4.4 Default Collation Declaration}}

    Galax \galaxversion\ does not support collations.

\subsection*{\ahref{\rawxqueryurl\#id-construction-declaration}{XQuery
Section 4.6 Construction Declaration}}

Galax \galaxversion\ does not support the construction declaration. 

\subsection*{\ahref{\rawxqueryurl\#id-default-ordering-decl}{XQuery
Section 4.8 Default Ordering Declaration}}

Galax \galaxversion\ does not support the default ordering declaration. 

\subsection*{\ahref{\rawxqueryurl\#id-schema-imports}{XQuery
Section 4.9 Schema Import}}

      Schema components in an imported schema are mapped into XQuery 
      types according to the mapping rules specified in the XQuery 1.0
      Formal Semantics (see below).

\section{XQuery 1.0 Formal Semantics}

  The XQuery 1.0 formal semantics defines the mapping of every XQuery
  expression into an expression in the XQuery core, and it defines the
  static and dynamic semantics of each core expression.  The formal
  semantics also defines how imported schemas are mapped into internal
  XQuery types.
  
  Galax \galaxversion\ implements the static and dynamic semantics of core
  expressions defined in the XQuery 1.0 Formal Semantics. 

\section{Functions and Operators}

  Galax \galaxversion\ supports most of the functions in the XQuery 1.0 and
  XPath 2.0 Functions and Operators document.  The signatures of
  supported functions are listed in \code{\${GALAXLIB}/pervasive.xq}.

  Galax \galaxversion\ does not support the following functions: 
\begin{verbatim}
  fn:id
  fn:idref
  fn:collection
\end{verbatim}

\subsection*{\ahref{\rawfandourl\#func-trace}{Functions and Operators Section 4 The Trace Function}}

    The \code{fn:trace} function emits its input sequence and message are to standard output. 

\subsection*{\ahref{\rawfandourl\#constructor-functions-for-user-defined-types}{Functions
and Operators Section 5.2 Constructor Functions for User-Defined
Types}} Galax \galaxversion\ does not support constructor functions
for user-defined types.

\subsection*{Functions and Operators Section 12 Functions and Operators on base64Binary and hexBinary}

 Galax \galaxversion\ does not support any functions on binary data.


\section{Use Cases}

  See \code{\${GALAXHOME}/usecases/STATUS}

\section{Galax \galaxversion\ extensions}

\subsection{Defining XQuery Types in the Query Prolog}

    Type values are available in a query by either importing a
    predefined XML schema using the \code{import schema} declaration in the
    query prolog or by defining XQuery types explicitly in the query prolog.

    Galax \galaxversion\ supports the definition of XQuery types in the query
    prolog using the internal type syntax defined in the XQuery 1.0
    Formal Semantics.  The grammar is provided here for reference:

\begin{verbatim}
    TypeDeclaration ::=    ("define" "element" QName "{" TypeDefn? "}")
                        |  ("define" "attribute" QName "{" TypeDefn? "}")
                        |  ("define" "type" QName "{" TypeDefn? "}")

    TypeDefn        ::=    TypeUnion 
                        |  TypeBoth 
                        |  TypeSequence 
                        |  TypeSimpleType 
                        |  TypeAttributeRef 
                        |  TypeElementRef 
                        |  TypeTypeRef 
                        |  TypeParenthesized 
                        |  TypeNone

    TypeUnion        ::= TypeDefn  "|"  TypeDefn
    TypeBoth         ::= TypeDefn  "&"  TypeDefn
    TypeSequence     ::= TypeDefn  ","  TypeDefn
    TypeSimpleType   ::= QName OccurrenceIndicator
    TypeAttributeRef ::= "attribute" NameTest ("{" TypeDefn? "}")? OccurrenceIndicator
    TypeElementRef   ::= "element" NameTest ("{" TypeDefn? "}")? OccurrenceIndicator
    TypeTypeRef      ::= "type" NameTest OccurrenceIndicator
    TypeParenthesized::= "(" TypeDefn? ")" OccurrenceIndicator
    TypeNone         ::= "none"
\end{verbatim}

\subsection{Galax specific functions}

    Galax-only functions are put in the Galax namespace
    (http://www.galaxquery.org), which is bound by default to the
    glx: prefix.

    See \code{\$GALAXHOME/lib/pervasive.xq} for a complete list of functions
    in the Galax namespace.




\chapter{Command Line Tools}
\label{sec:commandline}
\cutname{commandline.html}
Galax supports the following stand-alone command-line tools:

\begin{description}
\item[glx xquery] The Galax XQuery interpreter.               
\item[glx xml] XML document parser and validator. 
\item[glx xmlschema] XML Schema validator.  Outputs XML Schema in XQuery type system.
\item[glxd] The Galax network server.  Executes Galax query plans
  issued from remote client (Under development).
\end{description}

\section{\code{galax-run} : The Galax XQuery interpreter}
The simplest way to use Galax is by calling the 'glx xquery'
interpreter from the command line. The interpreter takes an XQuery
input file, evaluates it, and yields the result on standard output.

Usage: \code{glx xquery \emph{options} query.xq}

For instance, the following commands from the Galax home directory:
\begin{alltt}
%  echo "<two> { 1+1 } </two>" > test.xq
%  $(GALAXHOME)/bin/glx xquery test.xq
<two>2</two>
\end{alltt}
evaluates the simple query \verb|<two> { 1+1 } </two>| and prints the
XML value \verb|<two>2</two>|.

The query interpreter has eight processing stages:
   parsing an XQuery expression;
   normalizing an XQuery expression into an XQuery core expression;
   static typing of a core expression;
   rewriting a core expression;
   factorizing a core expression;
   compiling a core expression into a logical plan;
   selecting a physical plan from a logical plan;
   and evaluating a physical plan.

Parsing, factorization, and compilation are always enabled. By default, the other phases are:
\begin{alltt}
   -normalize on 
   -static off   
   -rewriting on 
   -optimization on
   -dynamic on 
\end{alltt}
By default, all result values (XML result, inferred type, etc.) are written
to standard output.

The command line options permit various combinations of phases,
printing intermediate expressions, and redirecting output to multiple
output files.  Here are the available options.  Default values are in
\code{code} font.

\begin{description}
\item[-help,--help]   Display this list of options
\end{description}

\subsection{Input options}
\label{sec:inputoptions}
\begin{description}
\item[-base-uri]  Sets the default base URI in the static context
\item[-var]  \code{x}=literal, Binds the global variable \code{x} to literal constant
\item[-doc]  \code{x}=filename, Binds the global variable \code{x} to document in filename
\item[-context-item] Binds the context item (``\code{.}'' variable) to the document in filename or '-' for stdin
\item[-context]  Load the context query from file or '-' for stdin
\end{description}

\begin{description}
\item[-xml-whitespace]  Preserves whitespace in XML documents [on/\code{off}]
\item[-xml-pic]  Preserves PI's and comments in XML documents [on/\code{off}]
% \item[-xquery-whitespace]  Preserves whitespace in XQuery expressions [on/\code{off}]
\end{description}

\subsection{Output options}
\begin{description}
\item[-serialize]  Set serialization kind [wf, canonical, or \code{xquery}]
% \item[-serialize-namespaces]  Set serialization of namespace nodes [strip/\code{preserve}]
\end{description}

\subsection{Evaluation options}
\begin{description}
\item[-dynamic]  Evaluation phase [\code{on}/off]

\item[-execute-normalized-expr]  All files are assumed to be normalized expressions for execution [{on}/\code{off}]
\item[-execute-logical-plan]  All files are assumed to be logical plans for execution [{on}/\code{off}]

\item[-print-xml]  Print XML result [\code{on}/off]
\item[-output-xml]  Output XML to result file

\item[-print-expr]  Print input expression [on/\code{off}]
\item[-output-expr]  Output expr to file.

\end{description}

\subsection{Normalization options}
\begin{description}
\item[-normalize]  Normalization phase [\code{on}/off]
\item[-print-normalized-expr]  Print normalized expression [on/\code{off}]
\item[-output-normalized-expr]  Output normalized expression to file.
\end{description}

\subsection{Static typing options}
\begin{description}
\item[-static] Static analysis phase [on/\code{off}]
\item[-typing] Static typing behavior [\code{none}, weak, or strong]

\item[-print-type]  Print type of expression [on/\code{off}]
\item[-output-type]  Output type to file.

\item[-print-typed-expr]  Print typed expression [on/\code{off}]
\item[-output-typed-expr]  Output typed expression to file.
\end{description}

\subsection{Rewriting options}
\begin{description}
\item[-rewriting] Rewriting phase (use with -static on) [\code{on}/off]
\item[-sbdo] Document-order/duplicate-elimination optimization
  behavior [remove, preserve, sloppy, tidy, \code{duptidy}]
\begin{description}
\item[remove] Remove all distinct-docorder (ddo) calls from the query
 plan. May result in faulty evaluation plans.
\item[preserve] Preserve all distinct-doc-order calls (no
 optimization).
\item[sloppy] Delay duplicate removal and sorting until the last
  step in the path expression. May cause the number of duplicates in
  intermediate results to increase exponentially since they are not
  removed. Also, subsequent steps are evaluated multiple times when
  duplicates are generated in a step.
\item[tidy] Removes all distinct-doc-order operations hat can be
 removed while maintaining duplicate-freeness and doc-order after each
 step.
\item[duptidy] Only sorts and removes duplicates after a step if duplicates are generated. Intermediate results can be out of document order.
\end{description}

\item[-print-rewritten-expr]  Print rewritten expression [on/\code{off}]
\item[-output-rewritten-expr]  Output rewritten expression to file.
\end{description}

\subsection{Factorization options}
\begin{description}
\item[-factorization] Factorization phase [\code{on}/off]

\item[-print-factorized-expr]  Print factorized expression [on/\code{off}]
\item[-output-factorized-expr]  Output factorized expression to file.
\end{description}

\subsection{Compilation options}
\begin{description}
\item[-print-comp-annotations]  Print compilation annotations [on/\code{off}]
\item[-print-logical-plan]  Print logical plan [on/\code{off}]
\item[-output-logical-plan]  Output logical plan to file.
\end{description}

\subsection{Optimization options}
\begin{description}
\item[-optimization] Optimization phase [\code{on}/off]
\item[-nested-loop-join] Turns off sort/hash joins and uses only nested-loop joins [\code{on}/off]


\item[-print-optimized-plan]  Print optimized plan [on/\code{off}]
\item[-output-optimized-plan]  Output optimized plan to file

\item[-print-physical-plan]  Print physical plan [on/\code{off}]
\item[-output-physical-plan]  Output physical algebraic plan to file

\end{description}

\subsection{Miscellaneous printing options}
\begin{description}
\item[-verbose]  Emit descriptive headers in output [on/\code{off}]
\item[-print-global]  Prints everything : prologs, exprs, etc.
\item[-output-all]  Output everything to file.

\item[-print-error-code]  Print only the error code instead of the full error message
\item[-output-err]  Redirect error output to file.

\item[-print-context-item]  Serializes the context item at the end of
  query evaluation
\item[-output-context-item]  Output the context item to the given file

\item[-print-annotations]  Print expression annotations [on/\code{off}]
\item[-print-prolog]  Print query prolog [on/\code{off}]

\item[-print-materialize]  Print whenever data materialization occurs [on/\code{off}]
\item[-force-materialized]  Force materialization of values in
  variables [on/\code{off}]

\end{description}

\subsection{Document projection options}
\begin{description}
\item[-projection]  Document projection behavior [\code{none}, standard, or optimized]
\item[-print-projection]  Prints the projection paths
\item[-output-projection]  Output the projections paths to a file.

\item[-print-projected-file]  Prints back the input document after projection
\item[-output-projected-file]  Output the input document after projection into file.

%\item[-pretty-print-module]  Pretty printing the whole module
%\item[-pretty-print]  Pretty printing the query to file 'filename\_pretty.xq'
\end{description}

\eat{\item[-update-hack-oi]  Set the OI flag on
\item[-update-hack-bi]  Set the BI flag on
\item[-update-hack-sbi]  Set the SBI flag on
}

\subsection{Miscellaneous options}
\begin{description}
\item[-version]  Prints the Galax version
\item[-debug]  Emit debugging [on/\code{off}]

\item[-monitor] Monitors memory and CPU consumption [on/\code{off}]
\item[-monitor-mem]  Monitors memory consumption [on/\code{off}]
\item[-monitor-time]  Monitors CPU consumption [on/\code{off}]
\item[-output-monitor]  Output monitor actibity to file

\item[-internal-encoding]  Set the input character encoding
  representation, e.g., \code{utf8, iso88591}.
\item[-output-encoding]  Set the output character encoding representation
\end{description}

\section{\code{glx xml}: XML parser and XML Schema validator}

\code{glx xml} parses an XML document and optionally validates the
document against an XML Schema.

Usage: \code{glx xml \emph{options} document.xml}

\begin{description}
\item[-help,--help]    Display this list of options

\item[-xml-whitespace]  Preserves whitespace in XML documents
\item[-xml-pic]  Preserves PI's and comments in XML documents

\item[-validate] Set validation on
\item[-xmlschema] Schema against which to perform validation

\item[-dm] Also builds the data model instance

\item[-monitor]  Monitors memory and CPU consumption
\item[-monitor-mem]  Monitors memory consumption
\item[-monitor-time]  Monitors CPU consumption
\item[-output-monitor]  Output monitor to file

\item[-serialize-namespaces]  Set serialization of namespace nodes  [strip/\code{preserve}]

\item[-serialize]  Set serialization kind [canonical, \code{wf}, or xquery]

\item[-print-error-code]  Print only the error code instead of the full error message
\item[-output-encoding]  Set the output encoding representation
\end{description}

\section{\code{glx xmlschema} : XML Schema validator}
\code{glx xmlschema} maps XML schemas in xmlschema(s) into XQuery type expressions.

Usage: \code{glx xmlschema \emph{options} schema.xsd}

\begin{description}
\item[-prefix] Namespace prefix
\item[-verbose]  Set printing to verbose
\item[-import-type]  Set XML Schema import
\item[-normalize-type]  Set XQuery type normalization

\item[-print-type]  Set printing ofXQuery type
\item[-print-normalized-type]  Set printing of normalized XQuery type

\item[-output-type] Output XQuery type in file
\item[-output-normalized-type] Output normalized XQuery type in file
\end{description}


\section{\code{glxd} : The Galax network server}

\code{glxd} is a server that allows Galax to be invoked over the  
network.

Usage: \code{glxd [\emph{options}] [query.xq]}

\begin{description}
\item[-port \emph{n}] Listen on port \emph{n}.  If the -port option is
   not used, the port defaults to 3324.
\item[-s \emph{dir}] Simulate the directory \emph{dir}.  The server
   will act as if it is a part of a virtual network specified by
   \emph{dir}.  Each file \emph{host.xq} in \emph{dir} defines a server
   \emph{host} in the virtual network.  The virtual network is
   implemented on the localhost at ports \emph{3324}, \emph{3325},
   \ldots.  Ports are assigned to the \emph{host.xq} files in
   lexicographic order.
\end{description}

The query file \emph{query.xq} should define a function named
local:main().  An XQuery program can get the result of local:main() on
\emph{host} by calling doc("dxq://host/").  If the server is using a
non-default port \emph{port}, then use doc("dxq://host:port/").

\subsection{Running simulations}
It is sometimes useful to simulate a network of Galax servers on a
single host.  The -s option makes this possible.  The way to set this
up is to create a directory with a query file for each simulated host.
For example, create a directory \emph{example} with query files
\emph{a.xq}, \emph{b.xq}, and \emph{c.xq}.  Each .xq file should
define a local:main() function.  Also, these files can refer to each
other's local:main() functions using doc("dxq://a/"), doc("dxq://b/"),
and doc("dxq://c/").  Then start up the three servers as follows:
\begin{verbatim}
     glxd -s example -port 3324
     glxd -s example -port 3325
     glxd -s example -port 3326
\end{verbatim}
\code{glxd} uses the -s option to find out what the virtual network
will look like: it will have hosts a, b, and c, operating on
non-virtual ports 3324, 3325, and 3326 on localhost.  The first
invocation of \code{galaxd} above uses port 3324, so it uses a.xq to
define its local:main() function.  Similarly, the second and third
invocations use b.xq and c.xq, respectively.



\chapter{Application Programming Interfaces (APIS)}
\label{sec:api}

% \textbf{\emph{NOTE: The C and Java APIs are disabled in 1.0.}}

The quickest way to learn how to use the APIs is as follows:

\begin{enumerate}
\item  Read Section~\ref{sec:apisupport} ``Galax API Support''.
\item  Read Section~\ref{sec:quickstart} ``Quick Start to the Galax APIs''.
\item  Read the example programs in the \code{galax/examples/} directory
  while reading  Section~\ref{sec:quickstart}. 
\end{enumerate}

Every Galax API has functions for:
\begin{itemize}
\item Converting values in the XQuery data model to/from values in
    the native programming language (O'Caml, C or Java);

\item Accessing values in XQuery data model from the native
  programming language;

\item Loading XML documents into the XQuery data model;

\item Creating and modifying the query evaluation environment (also
    known as the \term{dynamic context});

\item Evaluating queries given a dynamic context; and 

\item Serializing XQuery data model values as XML documents.
\end{itemize}
This chapter describes how to use each kind of functions. 

\section{Galax API Functionality}
\label{sec:apisupport}

  Galax currently supports application-program interfaces for the
  O'Caml, C, and Java programming languages. 

  All APIs support the same set of functions; only their names differ
  in each language API.  This file describes the API functions.  The
  interfaces for each language are defined in:

\begin{description}
\item[O'Caml] \cmd{\$GALAXHOME/lib/caml/galax.mli}
\item[C]      \cmd{\$GALAXHOME/lib/c/\{galax,galax\_util,galax\_types,itemlist\}.h}
\item[Java]   \cmd{\$GALAXHOME/lib/java/doc/*.html}
\end{description}

  If you use the C API, see Section~\ref{sec:cmemory} ``Memory Management in C API''.

  Example programs that use these APIs are in:
  
\begin{description}
\item[O'Caml] \cmd{\$GALAXHOME/examples/caml\_api}
\item[C]      \cmd{\$GALAXHOME/examples/c\_api}
\item[Java]   \cmd{\$GALAXHOME/examples/java\_api}
\end{description}

To try out the API programs, edit examples/Makefile.config to set up your environment, then
execute: \cmd{cd \$GALAXHOME/examples; make all}. 

This will compile and run the examples.  Each directory contains a "test"
program that exercises every function in the API and an "example"
programs that illustrates some simple uses of the API.

  The Galax query engine is implemented in O'Caml.  This means that
  values in the native language (C or Java) are converted into
  values in the XQuery data model (which are represented are by O'Caml
  objects) before sending them to the Galax engine.  The APIs provide
  functions for converting between native-language values and XQuery
  data-model values.

\subsection{Linking and Running}

  There are two kinds of Galax libraries: byte code and native code. 
  The C and Java libraries require native code libraries, and Java
  requires dynamically linked libraries.  Here are the libraries:

O'Caml libraries in \cmd{\$GALAXHOME/lib/caml}:
\begin{description}
\item[galax.cma]      Byte code
\item[galax.cmxa]     Native code
\end{description}


C libraries in \cmd{\$GALAXHOME/lib/c}:
\begin{description}
\item[libgalaxopt.a]   Native code, statically linked
\item[libgalaxopt.so]  Native code, dynamically linked
\end{description}


Java libraries in \cmd{\$GALAXHOME/lib/java}:
\begin{description}
\item[libglxoptj.so]  Native code, dynamically linked 
\end{description}

  Note that Java applications MUST link with a dynamically linked
  library and that C applications MAY link with a dynamically linked
  library.  

  For Linux users, set \cmd{LD\_LIBRARY\_PATH} to \cmd{\$GALAXHOME/lib/c:\$GALAXHOME/lib/java}.

  The Makefiles in \cmd{examples/c\_api} and \cmd{examples/java\_api} show how to
  compile, link, and run applications that use the C and Java APIs.

\section{Quick Start to using the APIs}
\label{sec:quickstart}

  The simplest API functions allow you to evaluate an XQuery statement
  in a string.  If the statement is an update, these functions return
  the empty list, otherwise if the statement is an Xquery expression,
  these functions return a list of XML values.

The example programs in
\code{\$(GALAXHOME)/examples/caml\_api/example.ml}, 
\code{\$(GALAXHOME)/examples/c\_api/example.c},
\code{\$(GALAXHOME)/examples/java\_api/Example.java} 
illustrate how to use these query evaluation functions. 

Galax accepts input (documents and queries) from files, string buffers, channels and HTTP, and
emits output (XML values) in files, string buffers, channels, and formatters. See
\code{\$(GALAXHOME)/lib/caml/galax\_io.mli}. 

All the evaluation functions require a processing context.  The
default processing context is constructed by calling the function \code{Processing\_context.default\_processing\_context()}:
\begin{alltt}
  val default_processing_context : unit -> processing_context
\end{alltt}

There are three ways to evaluate an XQuery statement:
\begin{alltt}
    val eval\_statement\_with\_context\_item : 
      Processing\_context.processing\_context -> Galax\_io.input\_spec -> 
        Galax\_io.input\_spec -> item list
\end{alltt}

    Bind the context item (the XPath "." expression) to the XML
    document in the resource named by the second argument, and
    evaluate the XQuery statement in the third argument.

\begin{alltt}
    val eval\_statement\_with\_context\_item\_as\_xml : 
      Processing\_context.processing\_context -> item -> 
        Galax\_io.input\_spec -> item list
\end{alltt}

    Bind the context item (the XPath "." expression) to the XML value
    in the second argument and evaluate the XQuery statement in
    the third argument. 
  
\begin{alltt}
  val eval\_statement\_with\_variables\_as\_xml : 
    Processing\_context.processing\_context -> 
      (string * item list) list -> 
        Galax\_io.input\_spec -> item list
\end{alltt}

    The second argument is a list of variable name and XML value pairs.
    Bind each variable to the corresponding XML value and evaluate the
    XQuery statement in the third argument.

  Sometimes you need more control over query evaluation, because, for
  example, you want to load XQuery libraries and/or main modules and
  evaluate statements incrementally.    The following two sections
  describe the API functions that provide finer-grained control.

\section{XQuery Data Model}

\subsection{Types and Constructors}

  In the XQuery data model, a value is a \term{sequence} (or list) of
  \term{items}.  An item is either an \term{node} or an \term{atomic value}.  An
  node is an \term{element}, \term{attribute}, \term{text}, \term{comment}, or
  \term{processing-instruction}.  An \term{atomic value} is one of the nineteen
  XML Schema data types plus the XQuery type \term{xs:untypedAtomic}.

  The Galax APIs provide constructors for the following data model
  values: 
\begin{itemize}
\item lists/sequences of items
\item element, attribute, text, comment, and processing instruction
    nodes
\item xs:string, xs:boolean, xs:int, xs:integer, xs:decimal, xs:float,
    xs:double, xs:anyURI, xs:QName, xs:dateTime, xs:date, xs:time,
  xs:yearMonthDuration, xs:dayTimeDuration, and xs:untypedAtomic.
\end{itemize}

\subsubsection{Atomic values}
  The constructor functions for atomic values take values in the
  native language and return atomic values in the XQuery data
  model.  For example, the O'Caml constructor:
\begin{alltt}
    val atomicFloat   : float -> atomicFloat
\end{alltt}
  takes an O'Caml float value (as defined in the Datatypes module) and
  returns a float in the XQuery data model.  Similarly, the C
  constructor: 

\begin{alltt}
    extern galax\_err galax\_atomicDecimal(int i, atomicDecimal *decimal);
\end{alltt}
  takes a C integer value and returns a decimal in the XQuery data
  model. 

\subsubsection{Nodes}
  The constructor functions for nodes typically take other data model
  values as arguments.  For example, the O'Caml constructor for
  elements: 
\begin{alltt}
    val elementNode : atomicQName * attribute list * node list * atomicQName -> element
\end{alltt}
  takes a QName value, a list of attribute nodes, a list of children
  nodes, and the QName of the element's type.  Simliarly, the C
  constructor for text nodes takes an XQuery string value:
\begin{alltt}
    extern galax\_err galax\_textNode(atomicString str, text *);
\end{alltt}

\subsubsection{Sequences}
  The constructor functions for sequences are language specific.  In
  O'Caml, the sequence constructor is simply the O'Caml list
  constructor.  In C, the sequence constructor is defined in
  galapi/itemlist.h as: 
\begin{alltt}
    extern itemlist itemlist\_cons(item i, itemlist cdr);
\end{alltt}

\subsection{Using XQuery data model values}
  The APIs are written in an "object-oriented" style, meaning that any
  use of a type in a function signature denotes any value of that type
  or a value derived from that type.  For example, the function
  \term{Dm\_functions.string\_of\_atomicvalue} takes any atomic value (i.e., xs\_string,
  xs\_boolean, xs\_int, xs\_float, etc.) and returns an O'Caml string
  value:
\begin{alltt}
    val string\_of\_atomicValue  : atomicValue -> string
\end{alltt}

  Similarly, the function \code{galax\_parent} in the C API takes any node value (i.e., an
  element, attribute, text, comment, or processing instruction node)
  and returns a list of nodes:
\begin{alltt}
    extern galax\_err galax\_parent(node n, node\_list *);
\end{alltt}

\subsection{Accessors}
  The accessor functions take XQuery values and return constituent
  parts of the value.  For example, the \code{children} accessor takes an
  element node and returns the sequence of children nodes contained in
  that element: 
\begin{alltt}
    val children : node -> node list      (* O'Caml *)
    extern galax\_err galax\_children(node n, node\_list *); /* C */
\end{alltt}

  The XQuery data model accessors are described in detail in 
  {\datamodelurl}. 

\subsection{Loading documents}
  Galax provides the \code{load\_document} function for loading documents.

  The \code{load\_document} function takes the name of an XML file in the
  local file system and returns a sequence of nodes that are the
  top-level nodes in the document (this may include zero or more
  comments and processing instructions and zero or one element node.)

\begin{alltt}
  val load\_document : Processing\_context.processing\_context -> 
    Galax\_io.input\_spec -> node list (* O'Caml *)
\end{alltt}

\begin{alltt}
  extern galax\_err galax\_load\_document(char* filename, node\_list *);
  extern galax\_err galax\_load\_document\_from\_string(char* string, node\_list *); 
\end{alltt}

\section{Query Evaluation}

  The general model for evaluating an XQuery expression or statement
  proceeds as follows (each function is described in detail below):
\begin{enumerate}
\item Create default processing context:

\cmd{let proc\_ctxt = default\_processing\_context() in}

\item Load Galax's standard library:

\cmd{let mod\_ctxt = load\_standard\_library(proc\_ctxt) in}

\item (Optionally) load any imported library modules:

\cmd{let library\_input = File\_Input "some-xquery-library.xq" in}
\cmd{let mod\_ctxt = import\_library\_module pc mod\_ctxt library\_input in}

\item (Optionally) load one main XQuery module:

\cmd{let (mod\_ctxt, stmts) = import\_main\_module mod\_ctxt (File\_Input "some-main-module.xq") in}

\item (Optionally) initialize the context item and/or global variables
     defined in application (i.e., external environment):

\cmd{let ext\_ctxt = build\_external\_context proc\_ctxt opt\_context\_item var\_value\_list in}
\cmd{let mod\_ctxt = add\_external\_context mod\_ctxt ext\_ctxt in}

\item Evaluate all global variables in module context:

\cmd{let mod\_ctxt = eval\_global\_variables mod\_ctxt}

     ** NB: This step is necessary if the module contains *any*
       global variables, whether defined in the XQuery module or
       defined externally by the application. **

\item Finally, evaluate a statement from the main module or one defined
     in the application or call some XQuery function defined in the
     module context:

     \cmd{let result = eval\_statement proc\_ctxt mod\_ctxt stmt in}

     \cmd{let result = eval\_statement\_from\_io proc\_ctxt mod\_ctxt
     (Buffer\_Input some-XQuery-statement) in}

     \cmd{let result = eval\_query\_function proc\_ctxt  mod\_ctxt
     "some-function" argument-values in}
\end{enumerate}

\subsection{Module context}
Every query is evaluated in a \term{module context}, which includes:
\begin{itemize}
\item the  built-in types, namespaces, and functions; 
\item the user-defined types, namespaces, and functions specified in
     any imported library modules; and
\item any additional context defined by the application (e.g., the values of
     the context item and any global variables).  
\end{itemize}

   The functions for creating a module context include:
\begin{alltt}
   val default\_processing\_context : unit -> processing\_context
\end{alltt}

      The default processing context, which just contains flags for
      controlling debugging, printing, and the processing phases.  You
      can change the default processing context yourself if you want
      to print out debugging info.

\begin{alltt}
   val load\_standard\_library : processing\_context -> module\_context
\end{alltt}
      Load the standard Galax library, which contains the built-in
      types, namespaces, and functions.

\begin{alltt}
val import\_library\_module : processing\_context -> 
  module\_context -> input\_spec -> module\_context
\end{alltt}

      If you need to import other library modules, this function
      returns the module\_context argument extended with the module
      in the second argument.

\begin{alltt}
val import\_main\_module    : processing\_context -> 
  module\_context -> input\_spec -> 
    module\_context * (Xquery\_ast.cstatement list)
\end{alltt}
      If you want to import a main module defined in a file, this
      function returns the module\_context argument extended with the
      main module in the second argument and a list of
      statements to evaluate.

   The functions for creating an external context (context item and
   global variable values):

\begin{alltt}
val build\_external\_context : processing\_context -> (item option) ->
  (atomicDayTimeDuration option) -> (string * item list) list ->  external\_context
\end{alltt}

     The external context includes an optional value for the context
     item (known as "."), the (optional) local timezone, and a list of
     variable name, item-list value pairs.

\begin{alltt}
val add\_external\_context : module\_context -> external\_context -> module\_context
\end{alltt}
This function extends the given module context with the external context.

\begin{alltt}
val eval\_global\_variables : processing\_context -> xquery\_module -> xquery\_module 
\end{alltt}
      This function evaluates the expressions for all (possibly
      mutually dependent) global variables.  It must be called before
      calling the eval\_* functions otherwise you will get an
      "Undefined variable" error at evaluation time.

   Analogous functions are defined in the C and Java APIs.
\subsection{Evaluating queries/expressions}

  The APIs support three functions for evaluating a query:
  \code{eval\_statement\_from\_io}, \code{eval\_statement}, and \code{eval\_query\_function}.

  \note{If the module context contains (possibly mutually
       dependent) global variables, the function \code{eval\_global\_variables} must be called before
       calling the eval\_* functions otherwise you will get an
       "Undefined variable" error at evaluation time.}

\begin{alltt}
val eval\_statement\_from\_io : processing\_context -> xquery\_module -> Galax\_io.input\_spec -> item list
\end{alltt}
       Given the module context, evaluates the XQuery statement  in
       the third argument.  If the statement is an XQuery expression,
       returns Some (item list); otherwise if the statement is an
       XQuery update, returns None (because update statements have
       side effects on the data model store, but do not return values).

\begin{alltt}
val eval\_statement 	   : processing\_context -> xquery\_module -> xquery\_statement -> item list
\end{alltt}
       Given the module context, evaluates the XQuery statement 

\begin{alltt}
val eval\_query\_function  : processing\_context -> xquery\_module -> string -> item list list -> item list
\end{alltt}
       Given the module context, evaluates the function with name in the
       string argument applied to the list of item-list arguments.
       \note{Each actual function argument is bound to one item list.}

   Analogous functions are defined in the C and Java APIs.

\subsection{Serializing XQuery data model values}

  Once an application program has a handle on the result of evaluating
  a query, it can either use the accessor functions in the API or it
  can serialize the result value into an XML document.  There are
  three serialization functions: \code{serialize\_to\_string},
  \code{serialize\_to\_output\_channel} and \code{serialize\_to\_file}. 

\begin{alltt}
val serialize  : processing\_context -> Galax\_io.output\_spec -> item list -> unit
\end{alltt}
    Serialize an XML value to the given galax output. 

\begin{alltt}
    val serialize\_to\_string : processing\_context -> item list -> string
\end{alltt}
Serializes an XML value to a string.

   Analogous functions are defined in the C and Java APIs.

\section{C API Specifics}

\subsection{Memory Management}
\label{sec:cmemory}
  The Galax query engine is implemented in O'Caml.  This means that
  values in the native language (C or Java) are converted into
  values in the XQuery data model (which represented are by O'Caml
  objects) before sending them to the Galax engine.  Similarly, the
  values returned from the Galax engine are also O'Caml values -- the
  native language values are "opaque handles" to the O'Caml values.

  All O'Caml values live in the O'Caml memory heap and are therefore
  managed by the O'Caml garbage collector.  The C API guarantees that
  any items returned from Galax to a C application will not be
  de-allocated by the O'Caml garbage collector, unless the C
  appliation explicitly frees those items, indicating that they are no
  longer accessible in the C appliation.  The C API provides two
  functions in galapi/itemlist.h for freeing XQuery item values: 
  
\begin{alltt}
extern void item\_free(item i);
\end{alltt}
      Frees one XQuery item value.
 
\begin{alltt}
extern void itemlist\_free(itemlist il);
\end{alltt}
      Frees every XQuery item value in the given item list.
  
\subsection{Exceptions}
  The Galax query engine may raise an exception in O'Caml, which must
  be conveyed to the C application.  Every function in the C API
  returns an integer error value : 
\begin{itemize}
\item     0 if no exception was raised or
\item    -1 if an exception was raised.
\end{itemize}

  The global variable galax\_error\_string contains the string value of
  the exception raised in Galax.  In future APIs, we will provide a
  better mapping between error codes and Galax exceptions

\section{Java API Specifics}

\subsection{General Info}

  The Galax query engine is implemented in O'Caml.  This means that
  values in the native language (C or Java) are converted into values
  in the XQuery data model (which represented are by O'Caml objects)
  before sending them to the Galax engine.

  The Java API uses JNI to call the C API, which in turn calls the
  O'Caml API (it's not as horrible as it sounds).  

  There is one class for each of the built-in XML Schema types
  supported by Galax and one class for each kind of node:

\begin{tabular}{lll}
    Atomic	 &  Node  &                  Item\\
    xsAnyURI     &  Attribute&\\
    xsBoolean    &  Comment\\
    xsDecimal    &  Element		\\
    xsDouble     &  ProcessingInstruction\\
    xsFloat      &  Text                 \\
    xsInt\\
    xsInteger\\
    xsQName\\
    xsString\\
    xsUntyped\\
\end{tabular}

There is one class for each kind of sequence:
\begin{itemize}
\item    ItemList
\item    AtomicList
\item    NodeList		    
\item    AttributeList    
\end{itemize}

There is one class for each kind of context used by Galax:
\begin{itemize}
\item    ExternalContext  
\item    ModuleContext	    
\item    ProcessingContext	    
\item    QueryContext	    
\end{itemize}

Finally, the procedures for loading documents, constructing new
contexts and running queries are in the \cmd{Galax} class.

\subsection{Exceptions}

  All Galax Java API functions can raise the exception class
  GalapiException, which must be handled by the Java application.

\subsection{Memory Management}

  All Java-C-O'Caml memory management is handled automatically in the
  Java API.

\section{Operation-System Notes}

  Currently, Galax is not re-entrant, which means multi-threaded
  applications cannot create multiple, independent instances of the
  Galax query engine to evaluate queries. 


\subsection{Windows}
\label{sec:api-notes-mingw}

The C API library \texttt{libgalaxopt.a,so} does not link properly
under MinGW.  A user reported that if you have the source
distribution, you can link directly with the object files in
\texttt{galapi/c\_api/*.o} and adding the library \texttt{-lasmrun} on
the command line works.

\eat{The C API seemed somewhat broken-- libgalaxopt.a contains libasmrun.a;objdump can't interpret libasmrun.a and linking to libgalaxopt.a reports thesymbols in libasmrun.a as unresolved; including libasmrun.a reports the Camlbootstrap symbols (i.e. from -output-obj in galax_wrap_opt.o inlibgalaxopt.a) as unresolved, etc.I'm not blocked by this moving forward-- linking directly togalapi/c_api/*.o with -lasmrun seems to work fine for me.}

\cutname{api.html}

\eat{
\chapter{Accessing and Storing XML}
\label{sec:documents}
\cutname{documents.html}
\section{Accessing XML Documents with \func{fn:doc()}}

\section{Storing and Accessing XML Documents with Jungle}
\label{sec:jungle}

\note{Documentation under construction}

To try out Jungle, make sure you have set up your environment as
described in Section~\ref{sec:install}, then execute following: 
\begin{alltt}
% cd \$(GALAXHOME)/examples/jungle
% make tests
\end{alltt}

\note{Don't forget in galax/examples/jungle director, to edit \cmd{jungle1.xq} and
    replace directory name by the path to your jungle directory, then execute: \cmd{make tests}.}

These commands will take a small XMark input document, create a Jungle
store, and run several example queries on the store. 

\subsection{\code{jungle-load} : The Jungle XML document loader}

Usage: \code{jungle-load \emph{options} input-xml-file}

\begin{description}
\item[-version]  Prints the Jungle loader version
\item[-help,--help]   Display this list of options
\end{description}

\begin{description}
\item[-store\_dir]  Directory Path where store is to be created  (default is current directory)
\item[-store\_name]  Logical name of the store (default is Jungle). 
\item[-buff\_size]  Size of the buffer to be used (default is 256KB).
\end{description}

In \cmd{\$(GALAXHOME)/examples/jungle}, execute following command to
build a Jungle store: 

\cmd{jungle-load -store\_dir tmp -store\_name XMark \$(GALAXHOME)/usecases/docs/xmark.xml}

After executing this command, the  \cmd{tmp} directory will contain:
\begin{alltt}
XMark-AttrIndex.db	  XMark-main.db      
XMark-Namespace.db	  XMark-Qname2QnameID.db	XMark-Text.db
XMark-FirstChildIndex.db  XMark-Metadata.db  
XMark-NextSiblingIndex.db	XMark-QnameID2Qname.db
\end{alltt}

\section{Implementing the Galax data model}


}

\chapter{For Galax Developers}
\label{sec:developers}
\cutname{developers.html}
\section{Galax Source Code Architecture}

The Galax source-code directories roughly correspond to each phase of
the query processor. (Put link to Jerome's tutorial presentation here)
  
The processing phases are: 

\paragraph{Document processing}
\begin{verbatim}
  Document Parsing =>  
 [Schema Normalization (below) =>]
    Validation => 
      Loading => 
        Evaluation (below)
\end{verbatim}

\paragraph{Schema processing}
\begin{verbatim}
  Schema Parsing => 
    Schema Normalization =>
      Validation (above)
      Static Typing (below)
\end{verbatim}

\paragraph{Query processing}
\begin{verbatim}
  Query Parsing => 
    Normalization => 
   [Schema Normalization (above) =>]
      Static Typing (optional phase) => 
        Rewriting => 
          Compilation => 
         [Loading (above) =>]
            Evaluation =>
              Serialization 
\end{verbatim}

\subsection{General}
Makefile
\begin{itemize}
\item  Main Makefile 
\end{itemize}

base/
\begin{itemize}
\item    Command-line argument parsing
\item    Global variables (conf.mlp)
\item    XQuery Errors
\item    String pools 
\item    XML Whitespace handling
\end{itemize}

ast/    
\begin{itemize}
\item  All ASTs: XQuery User \& Core, XQuery Type User \& Core
\item  Pretty printers for all ASTs
\end{itemize}

config/

monitor/	 
\begin{itemize}
\item  CPU \&/or memory monitoring of each processing phase
\end{itemize}

toplevel/	    
\begin{itemize}
\item    Main programs for command-line tools (see 'Generated executables'  below)
\end{itemize}

website/
\begin{itemize}
\item    Local copy of Galax web site
\end{itemize}

\subsection{Datamodel}

datatypes/  (*** Doug)
\begin{itemize}
\item    XML Schema simple datatypes  -- Lexers and basic operations 

\item    datatypes\_lexer.mll
    To learn about O'Caml lex, read:
      \ahrefurl{http://caml.inria.fr/ocaml/htmlman/manual026.html}
      Sections 12.1 and 12.2
    Other examples of lexers in lexing/*.mll

    We are going to extend this module to include lexer for: 
      xsd:date, xsd:time, xsd:dateTime, xs:yearMonthDuration, xs:dayTimeDuration
      (Skip Gregorian types for now, xsd:gDay, xsd:gMonth, etc)

\item  dateTime.ml,mli
    This module will implement the datatypes and basic operations
\end{itemize}

namespace/
\begin{itemize}
\item   XML Qualified Names (prefix:localname)
  -- Lexer and basic operations
  -- QName resolution prefix => URI 
\item   Names of builtin functions \& operators 
\end{itemize}

dm/	    (*** Doug)
\begin{itemize}
\item   Abstract data model interface for Nodes
\item   Concrete data model implementation for AtomicValues 
\end{itemize}

datamodel/
\begin{itemize}
\item   Main-memory implementation of abstract data model for Nodes
  (~ Document-Object Model or DOM)
\end{itemize}

jungledm/	 
\begin{itemize}
\item   Secondary storage implementation of Galax datamodel (Jungle)
\end{itemize}

physicaldm/	 
\begin{itemize}
\item   Physical data model 
\end{itemize}
streaming/
\begin{itemize}
\item   XML parser to untyped and typed SAX streams
\item   export datamodel to SAX stream 
\end{itemize}

\subsection{Processing Model}
procctxt/ 
\begin{itemize}
\item   Processing context contains all query-processor state:
\begin{itemize}
\item      Parse context
\item      Normalization context
\item      Static context
\item      Rewrite context
\item      Dynamic context
\end{itemize}
\end{itemize}

procmod/
\begin{itemize}
\item    Processing model dynamically "glues" together phases (controlled by
  command-line arguments or API)
\end{itemize}

\subsection{Query Parsing}
lexing/	    
\begin{itemize}
\item    Lexers for XQuery (excludes all simple datatypes)
\end{itemize}
parsing/	 
\begin{itemize}
\item    Parsing context
\item    Parsing phase
\end{itemize}

\subsection{Normalization}
normalization/
\begin{itemize}
\item    Normalization context
\item    Normalization phase (XQuery AST => XQuery Core AST)
\item    Overloaded functions
\end{itemize}

\subsection{Static Typing}

fsa/	    
\begin{itemize}
\item    Finite-state Automata for checking sub-typing relation
\end{itemize}
typing/
\begin{itemize}
\item    Static-typing context
\item    Static-typing phase
\end{itemize}

\subsection{Schema/Validation}
schema/
\begin{itemize}
\item   Schema-validation context
\item   Schema normalization phase (XML Schema => XQuery Core Types)
\item   Document validation phase
\item   Judgments(functions) for comparing XQuery types
\end{itemize}

\subsection{Rewriting}

cleaning/    
\begin{itemize}
\item   Logical optimization/rewriting phase 
\item   Sort-by-document order (DDO) optimization
\end{itemize}
rewriting/    
\begin{itemize}
\item   Generic AST rewriter
\end{itemize}

\subsection{Compilation}
compile/  
\begin{itemize}
\item   Compilation phase 
\end{itemize}
algebra/    
\begin{itemize}
\item   AST for compiled algebra 
\item   Dynamic context
\item   Implementations (dynamic) of most built-in functions \& operators
\end{itemize}

\subsection{Evaluation}
evaluation/
\begin{itemize}
\item   Evaluation phase
\end{itemize}
stdlib/ 
\begin{itemize}
\item   Static typing of built-in functions \& operators 
\item   Implementations (dynamic) of built-in functions fn:doc, fn:error
\item   Signatures of built-in functions \& operators (pervasive.xqp) 
     Corresponds to sections in \ahrefurl{http://www.w3c.org/TR/xpath-functions/}
\end{itemize}

\subsection{Serialization}
serialization/
\begin{itemize}
\item   Serialize SAX stream to XML document (in O'Caml formatter)
\end{itemize}

\subsection{Testing}

usecases/
\begin{itemize}
\item   Tests of XQuery Usecases
    Implements examples in \ahrefurl{http://www.w3.org/TR/xquery-use-cases/}
\end{itemize}
examples/
\begin{itemize}
\item   Tests of O'Caml, C \& Java APIs
\end{itemize}
regress/
\begin{itemize}
\item   Regression tests (needs separate xqueryunit/ CVS package)
\end{itemize}

\subsection{APIs}

galapi/   
\begin{itemize}
\item   O'Caml, C \& Java APIs to Galax processor
\end{itemize}

\subsection{External libraries \& tools}
tools/   

Required tools:
\begin{itemize}
\item   http
\item   pcre
\item   pxp-engine 
\item   netstring 
\end{itemize}

Optional supported tools:
\begin{itemize}
\item   Jungle
\end{itemize}

Optional unsupported tools:
\begin{itemize}
\item   glx\_curl
\item   jabber
\end{itemize}

\subsection{Extensions}
extensions/
\begin{itemize}
\item   apache
\item   jabber
\end{itemize}

\subsection{Experimental Galax extensions}

projection/ 
\begin{itemize}
\item   Document projection 
\end{itemize}

wsdl/    

wsdl\_usecases/
\begin{itemize}
\item   Web-service interfaces
\end{itemize}

\subsection{Documentation}
\begin{itemize}
\item Changes     
  Change log!! 
  Protocol: always document your changes in Changes file; use log
    entry as input message to 'cvs commit'

\item BUGS	    
  Out of date

\item LICENSE	 
\item README	 
\item STATUS	    
\item TODO	    
\end{itemize}


\subsection{Generated executables}
\begin{description}
\item[ocaml-galax] O'Caml top-level interpretor that loads Galax library.
               Usage:
               ocaml-galax -I \$(HOME)/Galax/lib/caml-devel

\item[glx] Complete XML processor
               For Usage:
               glx help
               See also: 
               all: rule in usecases/Makefile

\item[galax.a]       Library versions of Galax
\item[galax.cma]     byte code
\item[galax.cmxa]    machine code

\item[glx-map] Galax mapping tool.

\item[glxd] Galax network server.
\end{description}

Auxiliary research tools:
\begin{description}
\item[galax-mapwsdl]Imports/exports Galax queries as WSDL Web Services
\item[xquery2soap]

\item[galax-project] Takes XQuery query and figures out what fragments of
                documents are necessary to evaluate the query
\end{description}



\chapter{Release Notes}
\label{sec:releasenotes}
\cutname{release_notes.html}
\section{Galax 1.0 (March 2008)}

Galax version 1.0 implements the XQuery 1.0 Recommendation from
\xqueryrec\ (\xqueryurl) and the XQuery Update Facility 1.0 from
\ultfwd\ (\ultfurl).  It also implements XQueryP, an imperative
scripting language that extends XQuery with updates with mutable
variables, while loops, and sequential expressions (\xqueryp).

\begin{itemize}
\item Galax 1.0 must be built with O'Caml 3.10
\item Implementation of the latest XQuery Update Facility 1.0
(\ultfurl), including static typing.
\item Improved implementation of XQueryP.
\begin{itemize}
      \item Re-implementation of while loops.
      \item Fixed issues with scoping.
\end{itemize}
\item Improved support for modules.
\begin{itemize}
      \item Properly implemented nested imports.
      \item Added support for module interfaces.
\item Support for XQueryX trivial embedding.
\item Bug fixes:
\begin{itemize}
      \item Fixed problems with support for in-scope namespaces.
      \item Fixed problems with checking of cyclic variable declarations.
      \item Fixed several problems with the parser, using wrong lexical states inside constructors.
      \item Small bug fixes in support for the prolog.
\end{itemize}
\item Conformance results on XQTS 1.0.2:
\begin{tabular}{lrrrr}
  Feature & Pass & Fail & Total & Percent\\
  Minimal Conformance      &       14555 & 69 & 14637 & (99.4\%)
  Optional Features \\
    Static Typing Feature   &      46    & 0   & 46 \\
    Full Axis Feature        &     130  & 0   & 130\\
    Module Feature            &    32    & 0   & 32\\
\end{tabular}
\end{itemize}
\end{itemize}

\section{Galax 0.7.2 (February 2007)}

Galax version 0.7.2 is a minor release, and should be considered as a
beta release. Galax 0.7.2 implements the XQuery 1.0 Recommendation
from \xqueryrec.

This is a development release.  Notably static typing and
some of the new compiler optimizations are not fully tested.

\begin{itemize}
\item Compiler:
  \begin{itemize}
  \item We're still working on Join detection...A never-ending saga.
  \end{itemize}

\item Current testing results on XQTS 1.0.2:
\begin{tabular}{lrrrr}
  Feature & Galax Pass & Fail & Total & Percent\\
  Minimal Conformance	 &         14514 & 110 & 14637 & (99.1\%) \\
  Optional Features & \\
    Schema Import Feature&	  0   	& 0   & 174\\
    Schema Validation Feature&	  0   	& 0   & 25\\
    Static Typing Feature	&  46  	& 0   & 46\\
    Full Axis Feature	 	&  130 	& 0   & 130\\
    Module Feature	 	&  0 	& 0   & 32\\
    Trivial XML Embedding Feature &0 	& 0   & 4\\
\end{tabular}
\end{itemize}

\section{Galax 0.6.8 (August 2006)}

Galax version 0.6.8 is a minor release, and should be considered as a
beta release. Galax 0.6.8 implements the XQuery 1.0 candidate
recommendation working drafts from \xqueryversion.

This is a development source-only release.  Notably static typing and
some of the new compiler optimizations are not fully tested.

\begin{itemize}
\item Language: 
  \begin{itemize}
  \item More bug fixes to align with the \xqueryversion CR working
    drafts.
  \item Support for the XQuery! language is fixed (was broken in the
    previous release).
  \item Support for the XML update facility ({\ultfurl}).
  \item Support for the XQueryP extension to XQuery ({\xqueryp}).
  \end{itemize}
\item Compiler:
  \begin{itemize}
  \item Join detection is back! Some bug fixes and improvements to the
    robustness of the optimizer in the presence of rewritings.
  \end{itemize}
\end{itemize}

\section{Galax 0.6.5 (May 2006)}

Galax version 0.6.5 is a major release, and should be considered as an
alpha release. Galax 0.6.5 implements the XQuery 1.0 candidate
recommendation working drafts from \xqueryversion.

This is a development source-only release.  Notably static typing and
some of the new compiler optimizations are not fully tested.

\begin{itemize}
\item Language: 
  \begin{itemize}
  \item Alignment with the \xqueryversion CR working drafts. 
  \item Support for XML updates based on the XQuery! language [Alpha]   
  (See {\xquerybangurl}). 
  \item  Support for both strong and weak typing. [Alpha]. 
  \item Complete implementation of XQuery 1.0 Functions and Operators,
    notably full support for date and time, URI, and QName operations.
  \item Countless bug fixes...
  \end{itemize}
\item Environment:
  \begin{itemize}
  \item New configure script for automated configuration, compilation,
  and installation.  
  \item Galax is now a GODI package({\godiurl}).
  \item Added test harness for the W3C XQuery test-suite.
  Galax passes \textbf{\galaxpassestests} out of \textbf{\galaxtotaltests} tests on version
  \textbf{\xquerytestsuiteversion} of the XQuery 1.0 Test Suite.  The test
  suite contains a total of \textbf{\totaltests} tests. 
  \end{itemize}
\item Compiler:
  \begin{itemize}
  \item Support for a variety of physical algorithms for evaluating
  path expressions, including twig joins, staircase joins, and
  one-pass evaluation over XML token streams. [Alpha]
  \item Known problems:  Support for hash/sort joins is temporarily disabled.
  \end{itemize}
\end{itemize}

\section{Galax 0.5.0 (February 2005)}

Galax version 0.5 is a major release, and should be considered as an
alpha release. Galax 0.5.0 implements the XQuery 1.0 working draft
published in October 2004.

Among the most noticeable changes:

\begin{itemize}
\item Alignment with the XQuery 1.0, October 2004 working drafts.
\item A much faster XML parser, based on Gerd Stolpmann's PXP, fixing
  many XML 1.0 conformance bugs as well.
\item A completely new compiler, including a query optimizer that
  supports join optimizations and should deliver much better
  performances than the previous versions of Galax.
\item ``Document projection'' is finally part of the main release,
  allowing to process queries over large documents. (see the
  glx project command-line tool).
\item Improved support for sorting by document order and duplicate
  removal in the compiler.
\item The Windows port is back, based on the MinGW compilers.
\item New port for MacOS X.
\end{itemize}

Have contributed to this release:
  Mary Fern\'andez, Nicola Onose, Philippe Michiels, Christopher R\'e,
  J\'er\^ome Sim\'eon, Michael Stark.

\section{Galax 0.4.0 (August 2004)}

Galax version 0.4 is a major release, and should be considered as an
alpha release. Galax 0.4.0 implements the latest XQuery 1.0 working
draft published in July 2004. It contains many improvements from the
previous version, as well as new features.

Among the most noticeable improvements and new features: Galax now
comes bundled with Jungle, a simple native XML store. It now supports
XML Schema and "named typing". Finally, it contains some prototype
support for Web services.

Have contributed to this release:
  Mary Fern\'andez, Vladimir Gapeyev, Nicola Onose, Philippe Michiels,
  Doug Petkanics, Christopher R\'e, J\'er\^ome Sim\'eon, Avinash Vyas.

Main changes over the previous version are listed below.

Language changes:
\begin{itemize}
\item Support for the latest XQuery 1.0's specifications (July 2004
  Working Drafts).
\item Support for XML Schema 1.0: schema import, validate, named typing,
  sequence types and type tests.
\item Preliminary support for modules. Import module statements without
  recursion are supported. Details of the semantics will be fixed when
  issues around modules are addressed by the XML Query working group.
\item Support for dates and time.
\item Support for string regular expressions.
\end{itemize}

Environment changes:
\begin{itemize}
\item A new set of command-line tools replace the old ones:
    galax-run:       The XQuery execution engine
    galax-parse:     XML parsing and XML Schema validation
    galax-project:   To apply XML document projection
    galax-mapschema: To map XML Schema to the XQuery type system
    ocaml-galax:     The OCaml interpretor bundled with Galax
\item New command-line tools for Web services:
    galax-mapwsdl:   To map WSDL interfaces to an XQuery module
    xquery2soap:     To deploy an XQuery module as an apache Web
                     service.
\item Revisions to the Caml, C, C++ and Java APIs.
\end{itemize}

Architectural changes:
\begin{itemize}
\item A new extensible data model, making Galax easy to use over 'virtual'
  XML documents.
\item A completely new query compiler and evaluation engine that supports
  an hybrid SAX-tuple-tree algebra. The new compilation infrastructure
  should already show improved performances, although it performs
  little optimizations yet. Expect more work in this area in future
  versions of the system
\end{itemize}

New features:
\begin{itemize}
\item Alpha support for native XML storage with Jungle (on top of
  Sleepycat's BerkeleyDB).
\item Alpha support for calling Web services from within XQuery.
\end{itemize}

Portability:
\begin{itemize}
\item Added Makefile for Mac OSX in config/Makefile.osx.
\item Fixed numerous problems with Win32.
\end{itemize}


\section{Galax 0.3.5 (December 2003)}

This is a bug-fix release:
\begin{itemize}
\item Now compiles with OCaml 3.07.
\item Numerous bug fixes to the XML updates support.
\item Added glx:document-save() function allowing to save the result of a
query (notably useful in the case of an existing document that has
been updated).
\item Fixed bug in the release of Caml values in the C-API.
\item Fixed bug in pretty-printing of function application, and added
pretty-printing for the query prolog.
\item Fixed index bug in fn:substring and fn:translate.
\item Fixed serialization in canonical form.
\item Fixed bug in parsing of PIs in the document root.
\item Fixed bug in validation/casting of atomic values not dealing with
whitespace properly.
\item Fixed bug in key/keyref support introduced with the new API.
\item Fixed bug in parsing of DTD declarations with the PUBLIC keyword.
\end{itemize}


\section{Galax 0.3.1 (June 2003)}

Language extensions:
\begin{itemize}
\item Alpha support for XML updates!
\end{itemize}

Command line:
\begin{itemize}
\item External variables and context items can be bound from the
command-line.
\end{itemize}

API:
\begin{itemize}
\item Brand new, hopefully complete Caml, C, and Java APIs.
  Check them out!
\end{itemize}

Parsing:
\begin{itemize}
\item Switched for good to the SAX parser. glx:document-sax is removed.
\item Complete new support for character encodings. Now detects encoding
in XML declaration properly.  Support for UTF-16.
\end{itemize}


\section{Galax 0.3.0 (January 2003)}

Bugs
\begin{itemize}
\item All reported and fixed bugs are documented in 'Bugs' file.
\end{itemize}

Language:
Numerous changes to align with Nov. 2002 and upcoming Feb 2003 WDs.
\begin{itemize}
\item Grammar alignment:
  'as' SequenceType in type declarations, function signatures, typeswitch
\item Implements element \& attribute constructor semantics of Nov. 2002 WDs.
\item Added positional variables to FLWOR
\item Added support for order-by in FLWOR
\item Implemented complete semantics of path expressions, including
  document order and removing duplicates.
\item Implemented dynamic function dispatch, promotion of arguments to
  arithmetic operators
\item Transitive 'eq' operators 
\end{itemize}

Galax features:
\begin{itemize}
\item Command-line options for monitoring memory and CPU usage.
\end{itemize}

Data model:
\begin{itemize}
\item Updated terminology to align with WDs.
\end{itemize}

Function library:
\begin{itemize}
\item Changed xf: to fn: prefix
\item Added fn:error()
\item Added support for fn:base-uri() and fn:lang()
\item New semantics for fn:data() and fn:boolean()
\end{itemize}


\section{Galax 0.2.0 (October 2002)}

Parsing
\begin{itemize}
\item Fixed very large number of bugs. Support for entities and DTDs is
  still missing.
\item Support for ISO-8859-1 and UTF-8 character encodings.
\item Factorized XML and XQuery parsers. Results in more compact code,
  easier to improve and maintain.
\item Updated XQuery parser to align with latest grammar.
\item Alpha support for SAX-based parsing.
\end{itemize}

Data model:
\begin{itemize}
\item Support for node identity.
\item Fixed support for text nodes.
\end{itemize}

Language:
\begin{itemize}
\item Major revision based on August 16th 2002 working drafts.
\item Full support for XPath expressions. Notably XPath parent, ancestor,
  ancestor-or-self axis. See STATUS for some remaining deviations.
\item Support for node-identity related operations (is, isnot
  distinct-nodes, union, intersect, except, etc.).
\item Support for type promotion in function calls, arithmetics, etc.
\item Fixed many bugs in the semantics, through normalization.
\item Added dynamic semantics for type operations through (simplified)
  form of matching. Still some bugs there.
\end{itemize}

Namespaces:
\begin{itemize}
\item Fixed many bugs in namespace support and printing of namespaces.
\item Implemented support for default function namespace.
\end{itemize}

XML Schema:
\begin{itemize}
\item *Very* alpha support for XML Schema import and validation. Basic
  datatypes are now supported.
\end{itemize}

Type system:
\begin{itemize}
\item Updated type inference with the new language.
\item Made sure all expressions are type checked, necessary for
  optimization.
\item Fixed bugs in typing for typeswitch.
\end{itemize}

Function library:
\begin{itemize}
\item Most of F\&O functions are now implemented. Some limitations apply to
  XML Schema types not yet supported (notably date and time).
\end{itemize}

Optimizer:
\begin{itemize}
\item Support for simple query simplification.
\end{itemize}

Compilation:
\begin{itemize}
\item Removed dependency to the stdlib file.
\item Removed dependency to anything but OCaml compilers and standard unix
  tools.
\item Fixed compilation for both Cygwin and MinGW.
\end{itemize}

Tests:
\begin{itemize}
\item Added large number of regression tests.
\end{itemize}

Tools and interfaces:
\begin{itemize}
\item Removed Java API for now. Will be back soon.
\item Added very limited user-level api (Galapi) in Caml.
\item Changed galax command-lined interpretor. See the new syntax and
  options in ./README
\item Major revision of the pretty-printer for types and XQuery
  expressions. Added support for precedence in both cases.
\end{itemize}

Documentation:
\begin{itemize}
\item Added examples of calls to the Caml and C API's in ./example.
\end{itemize}



\chapter{General}
\label{sec:general}
\cutname{general.html}
\section{The Galax Team}

Contributors:

\begin{itemize}
\item Mary Fern\'andez (Lead)
\item Trevor Jim
\item Philippe Michiels
\item Kristi Morton
\item Nicola Onose
\item Chris Rath
\item Christopher R\'e
\item J\'er\^ome Sim\'eon (Lead)
\item Michael Stark
\end{itemize}

Alumni:

\begin{itemize}
\item Byron Choi
\item Vladimir Gapeyev
\item Am\'elie Marian
\item Douglas Petkanics
\item Gargi Sur
\item Avinash Vyas
\item Philip Wadler
\end{itemize}

\section{Feedback}

Feedback and bug reports can be sent by mail to:
{\galaxusers}

You can be subscribe to the Galax mailing list at:
{\galaxrequestusers}

You can report bugs at {\bugzillaurl}.

\section{Copyright and License}
\label{sec:license}

Galax version \galaxversion\ is distributed under the terms of the
LUCENT PUBLIC LICENSE VERSION 1.0 - see the LICENSE file for details.

\begin{alltt}
\input{../LICENSE}
\end{alltt}

\section{Bugs and Limitations}
\label{sec:bugs}
\section{Known Bugs and Limitations}

Galax's error messages are often uninformative.  We are working on
this.

Namespace declarations in input and output documents and in input
queries are not handled consistently.  We are working on this.

Although module declarations and module import statements are
supported, they are not well tested.

\begin{alltt}
\input{../BUGS}
\end{alltt}



\end{document}
